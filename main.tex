% 文章クラス:1段組のA4報告書(和文)
\documentclass[a4j,11pt]{jreport}
\usepackage{amssymb}
\usepackage{multirow} 
\usepackage{midpage}

% 文字のフォント指定
\usepackage{mathptmx}
\usepackage{newtxtext,newtxmath}
% \usepackage{titlesec}
% \titleformat*{\section}{\fontsize{12pt}{10pt}\selectfont\bf}
% \titleformat*{\subsection}{\fontsize{10pt}{10pt}\selectfont\bf}

\usepackage{titlesec}
\titleformat*{\section}{\fontsize{14pt}{0pt}\selectfont\bf}
\titleformat*{\subsection}{\fontsize{12pt}{0pt}\selectfont\bf}

\usepackage{setspace}
\usepackage{otf}%はしご高用
\setstretch{1.2}


%ヘッダーの設定
% \usepackage{fancyhdr}
% \usepackage{otf}
%  \pagestyle{fancy}
%     \lhead{\fontsize{10pt}{0pt}\selectfont{\ajRoman{2}}}
%      \chead{}
%     \rhead{}
%数式の設定%
\usepackage{amsmath}
%表のページ跨がせるよう%
\usepackage{longtable}
\usepackage{array}

\usepackage{comment}

% 余白の設定(上下左右:2cm)
\usepackage[margin=30truemm]{geometry}

% 図と表の設定
\usepackage[dvipdfmx]{graphicx}
\renewcommand{\figurename}{Fig. }
\renewcommand{\tablename}{Table}
\setlength\intextsep{10pt} % 図の余白
\setlength\textfloatsep{10pt} % 図の余白
\usepackage{here}

%tableやfigure内で図や表を出すためのコマンド
\makeatletter
\newcommand{\figcaption}[1]{\def\@captype{figure}\caption{#1}}
\newcommand{\tblcaption}[1]{\def\@captype{table}\caption{#1}}
\makeatother

% 追加パッケージ
\usepackage{comment}
\usepackage{subcaption}
\usepackage{url}
\usepackage{multicol}
\usepackage{CJKutf8}
\usepackage[dvipdfmx]{hyperref}
\usepackage{pxjahyper}
\usepackage{otf}
\usepackage{cleveref}
\crefname{figure}{Fig.}{Figs.}
\crefname{table}{Tab.}{Tabs.}
\usepackage{bm}
\usepackage{booktabs}
\usepackage{multirow}
\usepackage{siunitx}
\usepackage{minted}

% ドキュメント開始
\begin{document}
% ~~~表紙~~~
\begin{titlepage}
    \begin{center}
        \fontsize{16.060pt}{24.090pt}\selectfont{修士論文}\\
        \vspace{10.5pt}
        \fontsize{16.060pt}{24.090pt}\selectfont{意味的な画像概念のDNN学習過程における汎化性能について}\\
        \vspace{10.5pt}
        \fontsize{12.045pt}{18.067pt}\selectfont{On generalization performance under DNN learning process of semantic image concepts}\\
        \vspace{10.5pt}
        \fontsize{12.045pt}{18.067pt}\selectfont{\quad}\\
        \vspace{10.5pt}
        \fontsize{12.045pt}{18.067pt}\selectfont{\quad}\\
        \vspace{10.5pt}
        \fontsize{12.045pt}{18.067pt}\selectfont{\quad}\\
        \vspace{10.5pt}
        \fontsize{12.045pt}{18.067pt}\selectfont{\quad}\\
        \vspace{10.5pt}
        \fontsize{12.045pt}{18.067pt}\selectfont{\quad}\\
        \vspace{10.5pt}
        \fontsize{12.045pt}{18.067pt}\selectfont{\quad}\\
        \vspace{10.5pt}
        \fontsize{12.045pt}{18.067pt}\selectfont{\quad}\\
        \vspace{10.5pt}
        \fontsize{12.045pt}{18.067pt}\selectfont{\quad}\\
        \vspace{10.5pt}
        \fontsize{12.045pt}{18.067pt}\selectfont{\quad}\\
        \vspace{10.5pt}
        \fontsize{16.060pt}{24.090pt}\selectfont{\quad}\\
        \vspace{10.5pt}
        \fontsize{16.060pt}{24.090pt}\selectfont{東京電機大学大学院 システムデザイン工学研究科}\\
        \vspace{10.5pt}
        \fontsize{16.060pt}{24.090pt}\selectfont{情報システム工学専攻 修士課程}\\
        \vspace{10.5pt}
        \fontsize{16.060pt}{24.090pt}\selectfont{23AMJ03 岩瀬 俊}\\
        \vspace{10.5pt}
        \fontsize{16.060pt}{24.090pt}\selectfont{研究指導教員 \quad 教授 前田 英作}\\
        \vspace{10.5pt}
    \end{center}
\end{titlepage}

% 文章開始


\chapter*{要旨}
深層ニューラルネットワーク(DNN)を用いたエンドツーエンド学習は,コンピュータビジョン(CV)および自然言語処理(NLP)の諸タスクにおいて高い性能を実証している.しかしながら,分散表現に依存するDNNの解釈可能性は依然として限定的であり,ブラックボックスとして扱われることが多い.この透明性の欠如により,DNNが何を,いつ,どのように学習するのかという深層学習のメカニズムに対する深い理解が妨げられている.
複雑な画像認識タスクは一般的に,複数の意味的な画像概念を並行して学習することを伴い,異なる特徴が様々な時間スケールで学習される.従来研究では形状やテクスチャ特徴の学習が分析されてきたが,これらの概念は通常,与えられたデータセットに基づいて帰納的に定義されており,体系的な分析には制限があった.信頼性の高い分析を行うためには,適切かつ制御可能な難易度レベルと,それらを支持するための十分なデータを有する学習タスクの確立が不可欠である.
これらの課題に対し,本研究では「数字」と「色」という2つの解釈可能な概念に着目し,サンプル間に固有のノイズを含むEMNIST Digitsデータセットに色情報を付加することで,100クラスの分類タスクを構築した.その上で,標準的な条件下および追加的なラベルノイズ存在下におけるDNN学習プロセスの分析を実施した.
本研究の結果より,各概念の獲得難度に応じて学習のタイミングが異なること,また異なる概念の学習間に相互作用が存在することが明らかとなった.これらの知見は,深層学習における画像概念の学習プロセスに関する示唆を与えるものであり,画像ベースのタスクを超えた応用可能性を有するとともに,深層学習のダイナミクスの包括的な理解に寄与するものである.
\newpage

\chapter*{Abstract}

End-to-end learning using deep neural networks (DNNs) has demonstrated high performance across various computer vision (CV) and natural language processing (NLP) tasks. However, the interpretability of DNNs, which rely on distributed representations, remains limited, often rendering them as black boxes. This lack of transparency prevents a deeper understanding of the mechanics of deep learning, specifically regarding what, when, and how DNNs learn.
In contrast, complex image recognition tasks generally involve learning multiple semantic image concepts in parallel, with different features learned at varying time scales. While previous studies have analyzed the learning of shape and texture features, these concepts are typically defined inductively based on the given dataset, limiting systematic analysis.
For reliable analysis, it's crucial to establish learning tasks with appropriate, controllable difficulty levels and sufficient data to support them. To address these needs, we focused on two interpretable concepts—"numbers" and "colors"—and developed a classification task with 100 classes by adding color information to the EMNIST Digits dataset, which includes inherent noise across samples. We then analyzed the DNN learning process under standard conditions and with additional label noise.
Our results reveal that the timing of learning differs depending on the difficulty of acquiring each concept and that there is an interaction between learning different concepts. These findings offer insights into the learning process of image concepts in deep learning, with potential applications beyond image-based tasks, contributing to a broader understanding of deep learning dynamics.

\newpage

\tableofcontents
\newpage
\listoffigures
\newpage
\listoftables
\newpage

\chapter{序論}
深層ニューラルネットワーク(DNN)は,画像認識や自然言語処理など,幅広い分野において著しい性能向上を遂げてきた.
特に,ImageNetチャレンジにおける成功を契機に,DNNは視覚的タスクにおいて人間の能力を凌駕する成果を示している\cite{ILSVRC15}.
しかし,その性能の背後にある内部表現の形成メカニズムは未だ十分に解明されておらず,その「ブラックボックス」的な性質は重要な研究課題として残されている.
モデルの信頼性や透明性を高めるためには,学習プロセスにおける視覚的特徴の獲得メカニズムを明らかにする必要がある.
本研究は,DNNの視覚的特徴の学習プロセスにおいて,特徴獲得の順序や相互作用のメカニズムを明らかにすることを目的とする.
特に,学習過程における内部表現の動的変化を系統的に分析するため,新しい実験的枠組みを構築する.
視覚的特徴として「数字」と「色」に注目し,データセットを制御することで,学習過程を観察可能な環境を整備する.
本研究では,EMNIST Digitsデータセットに色情報を付加し,視覚的特徴の学習ダイナミクスを解析するための制御可能なデータセットを構築する.
これにより,視覚的特徴の学習順序,獲得速度,および特徴間の相互作用を詳細に分析する.
学習過程の評価には,モデルの精度を色や数字のみのエラー率に分解し,概念ごとの精度を観察する.
本研究の貢献は以下の3点に要約される.第一に,DNNの学習プロセスを解析するための制御可能な実験環境を構築した.
第二に,異なる視覚的特徴の学習順序および相互作用を定量的に評価し,特徴獲得メカニズムを明らかにした.
第三に,学習環境におけるノイズの影響を詳細に調査し,モデルのロバスト性と学習効率に関する重要な知見を得た.
これらの成果は,深層学習における内部表現の理解を深めるとともに,効率的かつ解釈可能な学習アルゴリズムの設計に向けた新たな指針を提供するものである.
\newpage
% \begin{figure}[t]
% \centering
% \includegraphics[width=1\columnwidth]{fig/fig1.pdf}
% \vspace{-40pt}
% \caption[ Flow of the analysis process comparing double descent with the learning process of image features.]{
% % 本稿で紹介する解析プロセスの流れ.我々は畳み込みニューラルネットワーク(CNN)を採用し,二重降下条件下で多様な画像認識タスクを訓練した.形状/テクスチャのバイアスとテストエラーの時間的変化を監視し,形状やテクスチャを解釈するモデルの能力を評価すると同時に,それらの相関関係を探った.
% %  Flow of the analysis process comparing double descent with the learning process of image features. We employed convolutional neural networks (CNNs) to train diverse image recognition tasks under double descent conditions. We monitored the temporal evolution of the shape/texture biases and test errors metrics assessing the capacity of the model to interpret shapes and textures while also exploring their correlation.
% }
% \label{fig:fig2}
% \end{figure}

\section{論文構成}
第1章「序論」では,深層学習の急速な発展とその多岐にわたる応用分野について概観し,特に画像認識分野における性能向上とその背景にある主要な技術的進展について述べた.
さらに,深層ニューラルネットワーク(DNN)の高い性能を支える内部表現の獲得メカニズムに関する既存の理論的枠組みと,
経験的に観測される挙動との間に見られるギャップを指摘し,本研究の対象とする課題の明確化を行った.
最後に,DNNの学習プロセスにおける視覚的特徴の獲得順序や相互作用の解明が,モデルの透明性向上や学習効率の改善に寄与することを示し,本研究の目的と貢献点を述べた.\par
第2章「先行研究」では,深層学習における視覚的特徴の獲得の研究がどのように進展してきたかを概観し,特に,画像認識における形状とテクスチャの重要性に関する先行研究を紹介する.
さらに,深層学習における重要な経験的に知られる二重降下現象に関する最近の研究成果を紹介し,深層学習の学習プロセスにおける特徴獲得のメカニズムに関する理論的知見と実験的結果を紹介する.\par
第3章「視覚的特徴の獲得を分析するためのデータセット」では,深層学習における,視覚的特徴の獲得メカニズムを紐解くための実験設定を提案し,その設定の理由と目的と述べる.\par
第4章「実験」では,実験手法を詳しく述べる,各種パラメータの変更による,仮説と目的を述べる.\par
第5章「結果」では,実験結果を示し,パラメータを変更することで得られた知見を述べる.\par
第6章「考察」では,実験結果についての考察を行う.パラメータを変化させることで得られた知見に関して考察をし,今回の実験において明らかになったことを述べる.\par
第7章「結論」では,本研究を総括する.\par
第8章「今後の展望」では,本研究で得られた知見から今後の方向性を示す.\par
「付録A」では,本文で紹介しきれなかった,基礎データを掲載する.\par
「付録B」では,学習コードの掲載場所を紹介する.\par
「付録C」では,6年間の学生生活における,対外成果紹介する.\par
\newpage
 
    
\chapter{先行研究}
\section{深層学習}
一般に,機械学習で使用されるモデルは決定木,サポートベクターマシン(SVM),ニューラルネットワークなどが存在する.決定木は得られた予測に対して,どの説明変数が影響したのかの判断が容易であり,説明可能性が高いことで知られている.
一方で,ニューラルネットワークは,パーセプトロンを筆頭に,層の増加やネットワークの複雑化が図られてきた.黎明期においては,非線形な問題をとけるように知見が盛り込まれたSVMや,生物が持つ視覚野の知見から提案されたネオコグニトロンなどの画期的な手法が提案されてきた.その中でも,ネオコグニトロンに端を発する,畳み込みニューラルネットワーク(CNN)は,LeNet\cite{LeNet}により,誤差逆伝播法が導入され,2010年代以降には,AlexNet\cite{AlexNet},VGGNet\cite{VGGNet},ResNet\cite{He:ResNet},と急速に進化を遂げてきた.
このような深層化されたニューラルネットワークは興味深い性質や振る舞いを示す.しかし,そのような性質がどのような機序によって引き起こされるかについての完全な合意はとられていない.

\section{二重降下現象}
機械学習において,モデルの性能はモデルの複雑性(例えば,パラメータ数)と深い関係があり,モデルのパラメータ数が不足することによるアンダーフィッティング(Underfitting)\cite{underfitting}や,過剰なパラメータによるオーバーフィッティング(Overfitting)\cite{overfitting}などの現象が知られている.モデルの複雑性が増すにつれて,初めは性能が向上し(アンダーフィッティングを克服),その後過剰な複雑性により性能が低下するとされていた.これはU字型のカーブ,いわゆるバイアス-バリアンス トレードオフ\cite{Rajnarayan2010}として知られている.

ところが近年発見されたDouble Descent\cite{Belkin_2019}と呼ばれている現象は,モデルの複雑性がさらに増すと,性能が再び向上する.つまり,最初のU字型のカーブ(アンダーフィッティングからオーバーフィッティングへの移行)の後,さらに複雑性が増加すると,新たな性能向上のフェーズが現れるのである.過剰パラメータを持つディープニューラルネットワークが,理論的にはオーバーフィッティングを起こすべきなのに,実際には優れた汎化性能を示す場合がある\cite{He:ResNet,ViT}.

このDouble Descentは,Belkinら\cite{Belkin_2019}によって決定木や二層のニューラルネットワークで確認され,その後,Nakkiranら\cite{nakkiran2021deep}が,ディープニューラルネットワーク(DNN)においても観察されること,学習エポック数の増加に対してもDouble Descentが起こることを示した.さらに,パラメータの枝刈りによるスパース性の増加に対してもDouble Descentが起こることが報告されている\cite{He}.パラメータ数,学習エポック数,スパース性の増加に伴って観察されるDouble Descentは,それぞれ,Model-wise Double Descent,Epoch-wise Double Descent,Sparse Double Descent と呼ばれている\cite{nakkiran2021deep,He}.

\subsection{Model-wise double descent}
Yangらは,バイアス-分散のトレードオフに関する古典的な理論を,広範な実験を通して再検討した\cite{Yang}.彼らは,分類理論が予測するようにバイアスが単調減少する一方で,分散は単峰性の挙動を示すことを発見した.このバイアスと分散の組み合わせは,3つの典型的なリスクカーブパターンを示唆しており,すでに報告されている多くの実験結果と一致している.また,Somepalliらは,新たな決定境界可視化手法を提案し,二重降下におけるエラーの悪化する領域において,決定境界が断片化していることを報告している\cite{Somepalli}.
さらに,Curthらは決定木などのBelkinらが二重降下を観察した条件において,2つの次元の軸によるU字のカーブの重ね合わせによって二重降下が起きると報告し,深層学習における二重降下においても,この観点は良い指針になることを示唆している\cite{Curth_NeurIPS2023}.

\subsection{Epoch-wise double descent}
統計的シミュレーションの結果から,学習過程における二重降下に関するいくつかの仮説が浮かび上がってきた.これらの仮説はデータの特徴に焦点を当てている.例えば,Stephensonらは,二重降下は遅いが有益な特徴によって起こると仮定し,理想的な線形モデルにおいてデータの主成分を除去することで二重降下の挙動を除去できることを示している\cite{Stephenson}.一方,Pezeshkiらは実験により,異なる時間スケールで学習された特徴が二重降下を引き起こすことを発見している\cite{Pezeshki}.さらに,Heckelらは,モデルの異なる部分が異なるエポックで学習することによる,複数のバイアスと分散の重複トレードオフが二重降下を引き起こすとしている\cite{Heckel}.そのうえで,層間で学習率を変えることで二重降下を緩和できることを実証している.

\subsection{Sparse double descent}
モデルのスパース性が高まるにつれて,つまり多くのパラメータがゼロまたは非常に小さくなるにつれ,まず性能の向上が見られる~\cite{He, SDD_VIT}.しかし,ある点を境に性能は低下する.さらにスパース性を高めると,性能は再び向上する.このことは,ネットワークの刈り込みのような方法で達成可能な適度なスパース性が,モデルのオーバーフィッティングを抑制し,汎化性能を向上させることを示唆している.また,枝刈り前のモデルの大きさに関わらず,枝刈り後の精度は一定である可能性が示唆されている~\cite{Arora_SDD}.

\section{画像認識における形状・テクスチャ}
Geirhosらは,ImageNetで学習したCNNが,分類のために特に画像のテクスチャを重視することを示した~\cite{Geirhos}.彼らは,相反する形状とテクスチャ情報を持つ画像をCNNに入力し,出力が形状ベースのラベルとテクスチャベースのラベルのどちらに一致するかをチェックした.この結果に基づいて,CNNが認識において形状とテクスチャのどちらを優先するかを分析した.一方,Islamらは,ニューロンの潜在表現に基づくモデルにおいて,形状とテクスチャのどちらを重視するかを定量的に判断する方法を提案した~\cite{Islam}.この方法によって,CNNがどの特徴に偏重するのかを定量的に分析することができる.さらに,Geらは人間の視覚系のモデル化を試み,Human Vision System (HVS)を開発した.HVSは,画像分類時にどの特徴(形状,テクスチャ,色など)が最も重要な役割を果たすかを定量的に評価可能である\cite{Ge}.

本研究は,画像理解と二重降下における CNN に関する先行研究を基礎としている.Islamらの手法を使用して,CNN学習中のテクスチャと形状情報に関する知識の獲得と二重降下現象との関係を明らかにすることを試みた.Islamらの手法を利用し,最終畳み込み層における形状・テクスチャそれぞれをエンコードするニューロン数を推定,推定した割合の値を形状・テクスチャ偏重度としている.
\newpage

\section{視覚的特徴の獲得を分析するための実験設定}

本章では,深層学習における,視覚的特徴の獲得メカニズムを紐解くための実験設定を提案し,その設定の理由と目的と述べる.




% \addcontentsline{toc}{chapter}{謝辞}
\chapter*{謝辞}
本修士論文は,東京電機大学システムデザイン工学研究科データ科学・機械学習研究室に所属し,前田英作教授の指導の下で執筆を行いました.3年半の間,研究指導にとどまらず,人生の道標となるような教えを賜りました指導教員の前田英作教授に厚く感謝申し上げます.
修士1年の1年間,副指導教員の前田 高志ニコラス准教授にはCTGの異常検知研究において,研究の方向性について貴重なご助言をいただきました.深く感謝申し上げます.
修士2年の1年間,本研究の遂行にあたり、副指導教員の川勝真喜准教授には修士2年次を通じて的確なご指導を賜りました。先生との議論を通じて既存の枠組みにとらわれない思考が培われ、研究の深化に大きく貢献していただきました。心より感謝申し上げます。

研究テーマの変更の上で,初期に有益な助言をいただき,その後の自身の研究において多大な貢献をしていただいたアルムナイの\UTF{9AD9}橋 秀弥氏に厚く感謝申し上げます.
研究室の前任事務岩本陽子氏,現任事務佐々木理央氏には,学会参加等の事務手続きにおいて多くの助力を賜り,修士生活を支えていただきました.深く感謝申し上げます.

酒造正樹客員教授には,論文指導等,研究生活において多くの助力を賜りました.厚く感謝申しあげます.

研究活動における数回の論文提出に際して,多くの助力をいただいた後輩の平本麗弥氏,小林慧音氏に深く感謝申し上げます.
研究生活において,議論,相談に付き合っていただいたり,疑問に答えていただいたデータ科学機械学習研究室の先輩,同期,後輩の皆様にも感謝を申し上げます.
産業技術総合研究所上級主任研究員片岡裕雄氏,東京工業大学横田理央教授,井上中順准教授,株式会社天地人 中村凌氏には,国際学会提出の際に,実験,執筆の双方に多くの助言と助力を賜りました.厚く御礼申し上げます.

最後に,私の生活を支えてくださった家族・友人に最大の感謝を申し上げます.

\newpage


% \renewcommand{\bibname}{参考文献}
\addcontentsline{toc}{chapter}{参考文献}  
% \begin{thebibliography}{99}
% \bibitem{sample}
% sample

% \end{thebibliography}
\bibliographystyle{ref/ieee_fullname.bst}
% \bibliographystyle{alpha}
\bibliography{ref/mybibfile}

% \appendix
% \chapter{基礎データ}
\label{付録}

\section{EMNIST Digitsデータセットの基礎データ}
\begin{figure}[H]
    \centering
    \includegraphics[width=\linewidth]{fig/color_varinace/0.pdf}
    \caption{$\sigma^2 = 0.0$の場合の各色クラスの色の例}
    \label{fig:variance_0}
\end{figure}

\begin{figure}[H]
    \centering
    \includegraphics[width=\linewidth]{fig/color_varinace/1000.pdf}
    \caption{$\sigma^2 = 10^3$の場合の各色クラスの色の例}
    \label{fig:variance_1000}
\end{figure}

\begin{figure}[H]
    \centering
    \includegraphics[width=\linewidth]{fig/color_varinace/3612.pdf}
    \caption{$\sigma^2 = 10^{3.5}$の場合の各色クラスの色の例}
    \label{fig:variance_3612}
\end{figure}

\begin{figure}[H]
    \centering
    \includegraphics[width=\linewidth]{fig/color_varinace/10000.pdf}
    \caption{$\sigma^2 = 10^4$の場合の各色クラスの色の例}
    \label{fig:variance_10000}
\end{figure}

\begin{figure}[H]
    \centering
    \includegraphics[width=\linewidth]{fig/distance_matrix_heatmap_by_color.pdf}
    \caption{100クラスのクラス間距離のヒートマップ(色,数字順)}
    \label{fig:distance_matrix_heatmap_by_color}
\end{figure}

\begin{figure}[H]
    \centering
    \includegraphics[width=\linewidth]{fig/distance_matrix_heatmap.pdf}
    \caption{100クラスのクラス間距離のヒートマップ(数字,色順)}
    \label{fig:distance_matrix_heatmap_by_digit}
\end{figure}

\begin{figure}[H]
    \centering
    \includegraphics[width=\linewidth]{fig/heatmap_ln/ln0.pdf}
    \caption{$\gamma=0.0$のときのモデルのチャネル幅とエポック数によるテストエラー率のヒートマップ.
    }
    \label{fig:modelwidth_heatmap_0}
\end{figure}

\begin{figure}[H]
    \centering
    \includegraphics[width=\linewidth]{fig/heatmap_ln/ln0.2.pdf}
    \caption{$\gamma=0.2$のときのモデルのチャネル幅とエポック数によるテストエラー率のヒートマップ.}
    \label{fig:modelwidth_heatmap_0.2}
\end{figure}

\begin{figure}[H]
    \centering
    \includegraphics[width=\linewidth]{fig/heatmap_ln/ln0.4.pdf}
    \caption{$\gamma=0.4$のときのモデルのチャネル幅とエポック数によるテストエラー率のヒートマップ.}
    \label{fig:modelwidth_heatmap_0.4}
\end{figure}

\begin{figure}[H]
    \centering
    \includegraphics[width=\linewidth]{fig/heatmap_ln/ln0.5.pdf}
    \caption{$\gamma=0.5$のときのモデルのチャネル幅とエポック数によるテストエラー率のヒートマップ.}
    \label{fig:modelwidth_heatmap_0.5}
\end{figure}

\begin{figure}[H]
    \centering
    \includegraphics[width=\linewidth]{fig/heatmap_ln/ln0.8.pdf}
    \caption{$\gamma=0.8$のときのモデルのチャネル幅とエポック数によるテストエラー率のヒートマップ.}
    \label{fig:modelwidth_heatmap_0.8}
\end{figure}

\newpage

\begin{figure}[H]
    \centering
    \includegraphics[width=\linewidth]{fig/kiyoritu_distribution_colored_EMNIST_Seed42_Var0_Corr0.5.pdf}
    \caption{Colored EMNISTデータセット($\sigma^2 = 0$)をPCAにかけたときの寄与率}
    \label{fig:modelwidth_heatmap_1}
\end{figure}

\begin{figure}[H]
    \centering
    \includegraphics[width=\linewidth]{fig/kiyoritu_distribution_colored_EMNIST_Seed42_Var10000_Corr0.5.pdf}
    \caption{Colored EMNISTデータセット($\sigma^2 = 10^4$)をPCAにかけたときの寄与率}
    \label{fig:modelwidth_heatmap_1}
\end{figure}


% % \chapter*{付録}
% \addcontentsline{toc}{chapter}{付録}
\chapter{学習プログラム}
\label{学習プログラム}
学習に利用したプログラムを示す.掲載したプログラムは,GitHub上で公開している.
\\
\url{https://github.com/dsml-lab/double_descent_and_shape_texture_bias}
% カスタムスタイルの定義
\lstdefinestyle{pythonstyle}{
    language=Python,
    basicstyle=\ttfamily\small,
    keywordstyle=\color{blue},
    stringstyle=\color{orange},
    commentstyle=\color{gray}\itshape,
    numbers=left,
    numberstyle=\tiny\color{gray},
    stepnumber=1,
    numbersep=5pt,
    backgroundcolor=\color{lightgray!10},
    frame=single,
    breaklines=true,
    showstringspaces=false,
    captionpos=b
}

% \begin{lstlisting}[style=pythonstyle, caption={メインコード}]
%     # combineのラベルノイズに関して、色・数字のラベル両方異なるラベルにする
%     # accuracyに関して、combineの正解率だけでなく、色・数字の正解率も出力する
%     # ラベルノイズの正解率とラベルノイズでない正解率を出力する
%     # 平均・分散も出力する
%     # 重みの加え方が均等
    
%     import os
%     import torch
%     import torch.nn as nn
%     import torch.optim as optim
%     import torch.nn.functional as f
%     import torch.backends.cudnn as cudnn
%     from torch.cuda.amp import autocast, GradScaler
%     from torch.utils.data.sampler import Sampler
%     import torchvision.transforms as transforms
%     from torch.utils.data import DataLoader, TensorDataset
%     from sklearn.metrics import classification_report
%     from sklearn.metrics import accuracy_score
%     import math
%     import torchvision.models as models
%     import numpy as np
%     import time
%     import wandb
%     import argparse
%     import random
%     import matplotlib.pyplot as plt
%     import csv 
%     import warnings
%     import gc
%     import gzip
%     # import models written by scratch
%     from model.cnn_2layers import CNN2Layer
%     from model.cnn_3layers import CNN3Layer
%     from model.cnn_4layers import CNN4Layer
%     from model.cnn_5layers import CNN5Layer
%     from model.cnn_8layers import CNN8Layer
%     from model.cnn_16layers import CNN16Layer
%     from model.resnet18 import ResNet18
%     # download datasets from pytorch
%     from torchvision import datasets
    
%     # Ignore warnings
%     warnings.filterwarnings("ignore")
%     # os.environ['CUDA_LAUNCH_BLOCKING'] = "1"
%     # os.environ['PYTORCH_CUDA_ALLOC_CONF'] = 'max_split_size_mb:128'
%     torch.backends.cudnn.benchmark = True
    
%     # settings
%     def parse_args():
%         """
%         Parse the command-line arguments.
        
%         Returns:
%             argparse.Namespace: The parsed command-line arguments.
%         """
%         arg_parser = argparse.ArgumentParser()
%         #set seed
%         arg_parser.add_argument("-seed", "--fix_seed", type=int, default=42)
        
%         #set model settings
%         arg_parser.add_argument("--model", type=str, choices=["cnn_2layers", "cnn_3layers", "cnn_4layers", "cnn_5layers", "cnn_8layers", "cnn_16layers", "resnet18"], help="モデルアーキテクチャの選択")
%         arg_parser.add_argument("-model_width", "--model_width", type=int, default=1)
%         arg_parser.add_argument("-epoch", "--epoch", type=int, default=1000)
        
%         #set dataset setting
%         arg_parser.add_argument("-datasets", "--dataset", type=str, choices=["mnist", "emnist", "emnist_digits", "cifar10", "cifar100", "tinyImageNet", "colored_emnist", "distribution_colored_emnist"], default="cifar10")
%         arg_parser.add_argument("-variance", "--variance", type=int, default=10000)
%         arg_parser.add_argument("-correlation", "--correlation", type=float, default=0.5)
%         arg_parser.add_argument("-label_noise_rate", "--label_noise_rate", type=float, default=0.0) 
%         arg_parser.add_argument("-gray_scale", "--gray_scale", action='store_true', help="グレースケールに変換するかどうか")
%         arg_parser.add_argument("-batch_size", "--batch_size", type=int, default=128, help="バッチサイズ")
%         arg_parser.add_argument("-img_size", "--img_size", type=int, default=32, help="画像サイズ")
%         arg_parser.add_argument("-target", "--target", type=str, choices=["color", "digit", "combined"], default='color', help="colored EMNISTのターゲットの指定:color or digit or combined")
        
%         # set optimizer setting
%         arg_parser.add_argument("-lr", "--lr", type=float, default=0.1, help="学習率")
%         arg_parser.add_argument("-optimizer", "--optimizer", type=str, choices=["sgd", "adam", "adamw", "rmsprop", "adagrad"], default="adam", help="最適化手法.adam were used in Nakkiran et al. (2019)")
%         arg_parser.add_argument("-momentum", "--momentum", type=float, default=0.9, help="モーメンタム")
        
%         #set loss function setting
%         arg_parser.add_argument("-loss", "--loss", type=str, choices=["cross_entropy", "focal_loss"], default="cross_entropy", help="損失関数")
        
%         # set device setting
%         arg_parser.add_argument("-gpu", "--gpu", type=int, default=0, help="GPU device ID")
%         arg_parser.add_argument("-num_workers", "--num_workers", type=int, default=4, help="データローダーの並列数")
        
%         # wandb setting
%         arg_parser.add_argument("-wandb", "--wandb", action='store_true',default=True ,help="wandbを使用するかどうか")
%         arg_parser.add_argument("-wandb_project", "--wandb_project", type=str, default="dd_scratch_models", help="wandbのプロジェクト名")
%         arg_parser.add_argument("--wandb_entity", type=str, default="dsml-kernel24", help="wandbのエンティティ名")
        
        
%         arg_parser.add_argument("-weight_noisy", "--weight_noisy", type=float, default=1.0, help="Weight for the loss of noisy samples")
%         arg_parser.add_argument("-weight_clean", "--weight_clean", type=float, default=1.0, help="Weight for the loss of clean samples")
%         return arg_parser.parse_args()
    
%     # Define a custom dataset that includes noise_info
%     class NoisyDataset(torch.utils.data.Dataset):
%         def __init__(self, dataset, noise_info):
%             self.dataset = dataset
%             self.noise_info = noise_info
    
%         def __len__(self):
%             return len(self.dataset)
    
%         def __getitem__(self, idx):
%             input, label = self.dataset[idx]
%             noise_label = self.noise_info[idx]
%             return input, label, noise_label
    
%     # Custom sampler to create balanced batches
%     class BalancedBatchSampler(Sampler):
%         def __init__(self, clean_indices, noisy_indices, batch_size, drop_last):
%             self.clean_indices = clean_indices
%             self.noisy_indices = noisy_indices
%             self.batch_size = batch_size
%             self.drop_last = drop_last
    
%             assert batch_size % 2 == 0, "Batch size must be even for balanced batches"
%             self.num_samples_per_class = batch_size // 2
    
%         def __iter__(self):
%             # Shuffle the indices
%             random.shuffle(self.clean_indices)
%             random.shuffle(self.noisy_indices)
    
%             # Calculate the number of batches
%             min_len = min(len(self.clean_indices), len(self.noisy_indices))
%             num_batches = min_len // self.num_samples_per_class
    
%             for i in range(num_batches):
%                 clean_batch = self.clean_indices[i * self.num_samples_per_class: (i + 1) * self.num_samples_per_class]
%                 noisy_batch = self.noisy_indices[i * self.num_samples_per_class: (i + 1) * self.num_samples_per_class]
%                 batch = clean_batch + noisy_batch
%                 random.shuffle(batch)
%                 yield batch
    
%             if not self.drop_last:
%                 # Handle remaining samples
%                 remaining_clean = self.clean_indices[num_batches * self.num_samples_per_class:]
%                 remaining_noisy = self.noisy_indices[num_batches * self.num_samples_per_class:]
    
%                 if len(remaining_clean) >= self.num_samples_per_class and len(remaining_noisy) >= self.num_samples_per_class:
%                     batch = remaining_clean[:self.num_samples_per_class] + remaining_noisy[:self.num_samples_per_class]
%                     random.shuffle(batch)
%                     yield batch
    
%         def __len__(self):
%             return len(self.clean_indices) // self.num_samples_per_class
    
%     # Set seeds
%     def set_seed(seed):
%         """
%         Set the seed for reproducibility.
    
%         Args:
%             seed (int): The seed value to set.
    
%         Returns:
%             None
%         """
%         random.seed(seed)
%         np.random.seed(seed)
%         torch.manual_seed(seed)
%         torch.cuda.manual_seed(seed)
%         torch.cuda.manual_seed_all(seed)
%         # For GPU determinism
%         torch.backends.cudnn.deterministic = True
%         torch.backends.cudnn.benchmark = False
    
%     # Set device
%     def set_device(gpu_id):
%         """
%         Sets the device for computation.
    
%         Args:
%             gpu_id (int): The ID of the GPU to use.
    
%         Returns:
%             torch.device: The selected device (GPU or CPU).
%         """
%         # Choose the GPU device if available, otherwise use CPU
%         device = torch.device("cuda:{}".format(gpu_id) if torch.cuda.is_available() else "cpu")
%         return device
    
%     def clear_memory():
%         torch.cuda.empty_cache()  # Clear CUDA cache
%         gc.collect()  # Force garbage collection
    
%     def apply_transform(x, transform):
%         transformed_x = []
%         for img in x:
%             img = transform(img)
%             transformed_x.append(img)
%         return torch.stack(transformed_x)
    
%     def load_datasets(dataset, target, gray_scale, args):
%         """
%         Load the specified dataset and apply transformations based on the dataset type and grayscale option.
        
%         Args:
%             dataset (str): The name of the dataset to load. Supported options are "mnist", "emnist", "cifar10", "cifar100", and "tinyImageNet".
%             gray_scale (bool): Flag indicating whether to convert the images to grayscale.
            
%         Returns:
%             tuple: A tuple containing the train dataset, test dataset, image size, and number of classes.
%         """
%         if dataset == "mnist":
%             transform = transforms.Compose([
%                 transforms.ToTensor(),
%                 transforms.Resize((32, 32)),
%                 transforms.Normalize((0.1307,), (0.3081,))
%             ])
%             train_dataset = datasets.MNIST(root='./data', train=True, download=True, transform=transform)
%             test_dataset = datasets.MNIST(root='./data', train=False, download=True, transform=transform)
%             imagesize = (32, 32)
%             num_classes = 10
%             in_channels = 1
%         elif dataset == "emnist":
%             transform = transforms.Compose([
%                 transforms.ToTensor(),
%                 transforms.Resize((32, 32)),
%                 transforms.Normalize((0.1307,), (0.3081,))
%             ])
%             train_dataset = datasets.EMNIST(root='./data', split='balanced', train=True, download=True, transform=transform)
%             test_dataset = datasets.EMNIST(root='./data', split='balanced', train=False, download=True, transform=transform)
%             imagesize = (32, 32)
%             num_classes = 47
%             in_channels = 1
%         elif dataset == "emnist_digits":
%             emnist_path = './data/EMNIST'
%             def load_gz_file(file_path, is_image=True):
%                 with gzip.open(file_path, 'rb') as f:
%                     if is_image:
%                         return np.frombuffer(f.read(), dtype=np.uint8, offset=16).reshape(-1, 28, 28)
%                     else:
%                         return np.frombuffer(f.read(), dtype=np.uint8, offset=8)
    
%             x_train = load_gz_file(os.path.join(emnist_path, 'emnist-digits-train-images-idx3-ubyte.gz'))
%             y_train = load_gz_file(os.path.join(emnist_path, 'emnist-digits-train-labels-idx1-ubyte.gz'), is_image=False)
%             x_test = load_gz_file(os.path.join(emnist_path, 'emnist-digits-test-images-idx3-ubyte.gz'))
%             y_test = load_gz_file(os.path.join(emnist_path, 'emnist-digits-test-labels-idx1-ubyte.gz'), is_image=False)
%             # 変換関数が必要な場合はここで定義
%             transform = transforms.Compose([
%                 transforms.ToPILImage(),  # Convert numpy array to PIL Image
%                 transforms.Resize((32, 32)),  # Same size as original, adjust if needed
%                 transforms.ToTensor()
%             ])
    
%             # Apply transformation
%             x_train_tensor = apply_transform(x_train, transform)
%             x_test_tensor = apply_transform(x_test, transform)
    
%             y_train_tensor = torch.tensor(y_train, dtype=torch.long)
%             y_test_tensor = torch.tensor(y_test, dtype=torch.long)
    
%             train_dataset = torch.utils.data.TensorDataset(x_train_tensor, y_train_tensor)
%             test_dataset = torch.utils.data.TensorDataset(x_test_tensor, y_test_tensor)
    
%             num_classes = 10  # Digits from 0 to 9
%             in_channels = 1  # Grayscale images
%             imagesize = (32, 32)  # Original image size
%         elif dataset == "colored_emnist":
%             # target: color or digit or combined
            
%             # Data augmentation
%             transform = transforms.Compose([
%                 transforms.ToPILImage(),  # Convert numpy array to PIL Image
%                 transforms.Resize((32, 32)),
%                 transforms.ToTensor(),
%                 transforms.Normalize((0.1307,), (0.3081,))
%             ])
            
%             if target == 'color':
%                 x_train = np.load('data/colored_EMNIST/x_train_colored.npy')
%                 y_train_colors = np.load('data/colored_EMNIST/y_train_colors.npy')
%                 x_test = np.load('data/colored_EMNIST/x_test_colored.npy')
%                 y_test_colors = np.load('data/colored_EMNIST/y_test_colors.npy')
                
%                 # Apply transformation
%                 x_train_tensor = apply_transform(x_train, transform)
%                 x_test_tensor = apply_transform(x_test, transform)
                
%                 y_train_tensor = torch.tensor(y_train_colors, dtype=torch.long)
%                 y_test_tensor = torch.tensor(y_test_colors, dtype=torch.long)
                
%                 train_dataset = torch.utils.data.TensorDataset(x_train_tensor, y_train_tensor)
%                 test_dataset = torch.utils.data.TensorDataset(x_test_tensor, y_test_tensor)
            
%             elif target == 'digit':
%                 x_train = np.load('data/colored_EMNIST/x_train_colored.npy')
%                 y_train_digits = np.load('data/colored_EMNIST/y_train_digits.npy')
%                 x_test = np.load('data/colored_EMNIST/x_test_colored.npy')
%                 y_test_digits = np.load('data/colored_EMNIST/y_test_digits.npy')
                
%                 # Apply transformation
%                 x_train_tensor = apply_transform(x_train, transform)
%                 x_test_tensor = apply_transform(x_test, transform)
                
%                 y_train_tensor = torch.tensor(y_train_digits, dtype=torch.long)
%                 y_test_tensor = torch.tensor(y_test_digits, dtype=torch.long)
                
%                 train_dataset = torch.utils.data.TensorDataset(x_train_tensor, y_train_tensor)
%                 test_dataset = torch.utils.data.TensorDataset(x_test_tensor, y_test_tensor)
                
%             elif target == 'combined':
%                 x_train = np.load('data/colored_EMNIST/x_train_colored.npy')
%                 y_train_combined = np.load('data/colored_EMNIST/y_train_combined.npy')
%                 x_test = np.load('data/colored_EMNIST/x_test_colored.npy')
%                 y_test_combined = np.load('data/colored_EMNIST/y_test_combined.npy')
                
%                 # Apply transformation
%                 x_train_tensor = apply_transform(x_train, transform)
%                 x_test_tensor = apply_transform(x_test, transform)
                
%                 y_train_tensor = torch.tensor(y_train_combined, dtype=torch.long)
%                 y_test_tensor = torch.tensor(y_test_combined, dtype=torch.long)
                
%                 train_dataset = torch.utils.data.TensorDataset(x_train_tensor, y_train_tensor)
%                 test_dataset = torch.utils.data.TensorDataset(x_test_tensor, y_test_tensor)
                
%             num_classes = 10 if target in ['color', 'digit'] else 100
%             in_channels = 3
%             imagesize = (32, 32)
%         elif dataset == "distribution_colored_emnist":
%             # target: color or digit or combined
%             seed = args.fix_seed
%             variance = args.variance
%             correlation = args.correlation
%             # Data augmentation
%             transform = transforms.Compose([
%                 transforms.ToPILImage(),  # Convert numpy array to PIL Image
%                 transforms.Resize((32, 32)),
%                 transforms.ToTensor(),
%                 transforms.Normalize((0.1307,), (0.3081,))
%             ])
            
%             if target == 'color':
%                 x_train = np.load(f'data/distribution_colored_EMNIST_Seed42_Var{variance}_Corr{correlation}/x_train_colored.npy')
%                 y_train_colors = np.load(f'data/distribution_colored_EMNIST_Seed42_Var{variance}_Corr{correlation}/y_train_colors.npy')
%                 x_test = np.load(f'data/distribution_colored_EMNIST_Seed42_Var{variance}_Corr{correlation}/x_test_colored.npy')
%                 y_test_colors = np.load(f'data/distribution_colored_EMNIST_Seed42_Var{variance}_Corr{correlation}/y_test_colors.npy')
                
%                 # Apply transformation
%                 x_train_tensor = apply_transform(x_train, transform)
%                 x_test_tensor = apply_transform(x_test, transform)
                
%                 y_train_tensor = torch.tensor(y_train_colors, dtype=torch.long)
%                 y_test_tensor = torch.tensor(y_test_colors, dtype=torch.long)
                
%                 train_dataset = torch.utils.data.TensorDataset(x_train_tensor, y_train_tensor)
%                 test_dataset = torch.utils.data.TensorDataset(x_test_tensor, y_test_tensor)
            
%             elif target == 'digit':
%                 x_train = np.load(f'data/distribution_colored_EMNIST_Seed42_Var{variance}_Corr{correlation}/x_train_colored.npy')
%                 y_train_digits = np.load(f'data/distribution_colored_EMNIST_Seed42_Var{variance}_Corr{correlation}/y_train_digits.npy')
%                 x_test = np.load(f'data/distribution_colored_EMNIST_Seed42_Var{variance}_Corr{correlation}/x_test_colored.npy')
%                 y_test_digits = np.load(f'data/distribution_colored_EMNIST_Seed42_Var{variance}_Corr{correlation}/y_test_digits.npy')
                
%                 # Apply transformation
%                 x_train_tensor = apply_transform(x_train, transform)
%                 x_test_tensor = apply_transform(x_test, transform)
                
%                 y_train_tensor = torch.tensor(y_train_digits, dtype=torch.long)
%                 y_test_tensor = torch.tensor(y_test_digits, dtype=torch.long)
                
%                 train_dataset = torch.utils.data.TensorDataset(x_train_tensor, y_train_tensor)
%                 test_dataset = torch.utils.data.TensorDataset(x_test_tensor, y_test_tensor)
                
%             elif target == 'combined':
%                 x_train = np.load(f'data/distribution_colored_EMNIST_Seed42_Var{variance}_Corr{correlation}/x_train_colored.npy')
%                 y_train_combined = np.load(f'data/distribution_colored_EMNIST_Seed42_Var{variance}_Corr{correlation}/y_train_combined.npy')
%                 x_test = np.load(f'data/distribution_colored_EMNIST_Seed42_Var{variance}_Corr{correlation}/x_test_colored.npy')
%                 y_test_combined = np.load(f'data/distribution_colored_EMNIST_Seed42_Var{variance}_Corr{correlation}/y_test_combined.npy')
                
%                 # Apply transformation
%                 x_train_tensor = apply_transform(x_train, transform)
%                 x_test_tensor = apply_transform(x_test, transform)
                
%                 y_train_tensor = torch.tensor(y_train_combined, dtype=torch.long)
%                 y_test_tensor = torch.tensor(y_test_combined, dtype=torch.long)
                
%                 train_dataset = torch.utils.data.TensorDataset(x_train_tensor, y_train_tensor)
%                 test_dataset = torch.utils.data.TensorDataset(x_test_tensor, y_test_tensor)
        
%             num_classes = 10 if target in ['color', 'digit'] else 100
%             in_channels = 3
%             imagesize = (32, 32)
%         elif dataset == "cifar10":
%             transform = transforms.Compose([
%                 transforms.RandomCrop(32, padding=4),
%                 transforms.RandomHorizontalFlip(),
%                 transforms.ToTensor(),
%                 transforms.Normalize((0.4914, 0.4822, 0.4465), (0.2023, 0.1994, 0.2010))
%             ])
%             train_dataset = datasets.CIFAR10(root='./data', train=True, download=True, transform=transform)
%             test_dataset = datasets.CIFAR10(root='./data', train=False, download=True, transform=transform)
%             imagesize = (32, 32)
%             num_classes = 10
%             in_channels = 3
%         elif dataset == "cifar100":
%             transform = transforms.Compose([
%                 transforms.RandomCrop(32, padding=4),
%                 transforms.RandomHorizontalFlip(),
%                 transforms.ToTensor(),
%                 transforms.Normalize((0.5071, 0.4867, 0.4408), (0.2675, 0.2565, 0.2761))
%             ])
%             train_dataset = datasets.CIFAR100(root='./data', train=True, download=True, transform=transform)
%             test_dataset = datasets.CIFAR100(root='./data', train=False, download=True, transform=transform)
%             imagesize = (32, 32)
%             num_classes = 100
%             in_channels = 3
%         elif dataset == "tinyImageNet":
%             transform = transforms.Compose([
%                 transforms.RandomCrop(64, padding=4),
%                 transforms.RandomHorizontalFlip(),
%                 transforms.ToTensor(),
%                 transforms.Normalize((0.4802, 0.4481, 0.3975), (0.2302, 0.2265, 0.2262))
%             ])
%             train_dataset = datasets.ImageFolder(root='./data/tiny-imagenet-200/train', transform=transform)
%             test_dataset = datasets.ImageFolder(root='./data/tiny-imagenet-200/val', transform=transform)
%             imagesize = (64, 64)
%             num_classes = 200
%             in_channels = 3
%         elif dataset == "distribution_to_normal":
%             # target: color or digit or combined
%             seed = args.fix_seed
%             variance = args.variance
%             correlation = args.correlation
%             # Data augmentation
%             transform = transforms.Compose([
%                 transforms.ToPILImage(),  # Convert numpy array to PIL Image
%                 transforms.Resize((32, 32)),
%                 transforms.ToTensor(),
%                 transforms.Normalize((0.1307,), (0.3081,))
%             ])
%             if target == 'combined':
%                 x_train = np.load(f'data/distribution_colored_EMNIST_Seed42_Var{variance}_Corr{correlation}/x_train_colored.npy')
%                 y_train_combined = np.load(f'data/distribution_colored_EMNIST_Seed42_Var{variance}_Corr{correlation}/y_train_combined.npy')
%                 x_test = np.load('data/colored_EMNIST/x_test_colored.npy')
%                 y_test_combined = np.load('data/colored_EMNIST/y_test_combined.npy')
                
%                 # Apply transformation
%                 x_train_tensor = apply_transform(x_train, transform)
%                 x_test_tensor = apply_transform(x_test, transform)
                
%                 y_train_tensor = torch.tensor(y_train_combined, dtype=torch.long)
%                 y_test_tensor = torch.tensor(y_test_combined, dtype=torch.long)
                
%                 train_dataset = torch.utils.data.TensorDataset(x_train_tensor, y_train_tensor)
%                 test_dataset = torch.utils.data.TensorDataset(x_test_tensor, y_test_tensor)
        
%             num_classes = 10 if target in ['color', 'digit'] else 100
%             in_channels = 3
%             imagesize = (32, 32)
%         else:
%             raise ValueError("Invalid dataset name")
            
    
%         if gray_scale:
%             transform = transforms.Compose([
%                 transforms.Grayscale(),
%                 transforms.ToTensor(),
%                 transforms.Normalize((0.1307,), (0.3081,))
%             ])
%             train_dataset.transform = transform
%             test_dataset.transform = transform  
    
%         return train_dataset, test_dataset, imagesize, num_classes, in_channels
    
%     def load_models(in_channels, args, img_size, num_classes):
%         if args.model == "cnn_2layers":
%             model = CNN2Layer(in_channels, num_classes, args.model_width, img_size)
%         elif args.model == "cnn_3layers":
%             model = CNN3Layer(in_channels, num_classes, args.model_width, img_size)
%         elif args.model == "cnn_4layers":
%             model = CNN4Layer(in_channels, num_classes, args.model_width, img_size)
%         elif args.model == "cnn_5layers":
%             model = CNN5Layer(in_channels, num_classes, args.model_width, img_size)
%         elif args.model == "cnn_8layers":
%             model = CNN8Layer(in_channels, num_classes, args.model_width, img_size)
%         elif args.model == "cnn_16layers":
%             model = CNN16Layer(in_channels, num_classes, args.model_width, img_size)
%         elif args.model == "resnet18":
%             model = models.resnet18(num_classes=num_classes)
%         else:
%             raise ValueError("Invalid model name.")
%         return model
    
%     # Modify add_label_noise to return indices of noisy samples
%     def add_label_noise(targets, label_noise_rate, num_digits, num_colors):
%         noisy_targets = targets.clone()
%         num_noisy = int(label_noise_rate * len(targets))
%         noisy_indices = torch.randperm(len(targets))[:num_noisy]
%         noise_info = torch.zeros(len(targets), dtype=torch.int)  # Initialize as clean
    
%         if num_digits == 10 and num_colors == 1:
%             for idx in noisy_indices:
%                 original_label = targets[idx].item()
%                 new_label = random.randint(0, num_digits - 1)
%                 while new_label == original_label:
%                     new_label = random.randint(0, num_digits - 1)
%                 noisy_targets[idx] = new_label
%                 noise_info[idx] = 1  # Mark as noisy
    
%         elif num_digits == 10 and num_colors == 10:
%             for idx in noisy_indices:
%                 original_label = targets[idx].item()
%                 original_digit = original_label // num_colors
%                 original_color = original_label % num_colors
    
%                 new_digit = random.randint(0, num_digits - 1)
%                 new_color = random.randint(0, num_colors - 1)
%                 new_label = new_digit * num_colors + new_color
%                 while new_label == original_label:
%                     new_digit = random.randint(0, num_digits - 1)
%                     new_color = random.randint(0, num_colors - 1)
%                     new_label = new_digit * num_colors + new_color
    
%                 noisy_targets[idx] = new_label
%                 noise_info[idx] = 1  # Mark as noisy
    
%         return noisy_targets, noise_info
    
%     def train_model(model, train_loader, optimizer, criterion, weight_noisy, weight_clean, device, num_colors, num_digits):
%         """
%         Training function with comprehensive metrics tracking.
        
%         Args:
%             model: The neural network model
%             train_loader: DataLoader for training data
%             optimizer: The optimizer
%             criterion: Loss function (expected to support reduction='none')
%             weight_noisy: Weight for noisy samples
%             weight_clean: Weight for clean samples
%             device: Device to run the training on
%             num_colors: Number of color classes
%             num_digits: Number of digit classes
            
%         Returns:
%             dict: Dictionary containing all training metrics
%         """
%         model.train()
%         running_loss = 0.0
%         total_samples = 0
%         correct_total = 0
    
%         # Initialize counters for noisy and clean samples
%         correct_noisy = 0
%         total_noisy = 0
%         correct_clean = 0
%         total_clean = 0
    
%         # Initialize counters for digits and colors
%         correct_digit_total = 0
%         correct_color_total = 0
    
%         correct_digit_noisy = 0
%         correct_color_noisy = 0
%         correct_digit_clean = 0
%         correct_color_clean = 0
    
%         # Lists for loss tracking
%         loss_values = []
%         loss_values_noisy = []
%         loss_values_clean = []
    
%         # Ensure criterion returns per-sample losses
%         criterion.reduction = 'none'
    
%         for inputs, labels, noise_labels in train_loader:
%             try:
%                 # Move data to device
%                 inputs = inputs.to(device, non_blocking=True)
%                 labels = labels.to(device, non_blocking=True)
%                 noise_labels = noise_labels.to(device, non_blocking=True)
    
%                 # Zero the parameter gradients
%                 optimizer.zero_grad()
    
%                 # Forward pass
%                 outputs = model(inputs)
%                 _, predicted = torch.max(outputs.data, 1)
                
%                 # Calculate batch size and update total samples
%                 batch_size = labels.size(0)
%                 total_samples += batch_size
    
%                 # Get indices for noisy and clean samples
%                 idx_noisy = (noise_labels == 1)
%                 idx_clean = (noise_labels == 0)
    
%                 # Count noisy and clean samples in batch
%                 num_noisy = idx_noisy.sum().item()
%                 num_clean = idx_clean.sum().item()
    
%                 # Compute per-sample losses
%                 per_sample_loss = criterion(outputs, labels)
    
%                 # Calculate weights for the batch
%                 total_weight = weight_clean + weight_noisy
%                 weights = torch.zeros_like(per_sample_loss, device=device)
                
%                 if num_noisy == 0:  # All clean samples
%                     weights = torch.ones_like(per_sample_loss, device=device) * (weight_clean / total_weight) * 2
%                 elif num_clean == 0:  # All noisy samples
%                     weights = torch.ones_like(per_sample_loss, device=device) * (weight_noisy / total_weight) * 2
%                 else:  # Mixed batch
%                     weights[idx_noisy] = (weight_noisy / total_weight) * 2
%                     weights[idx_clean] = (weight_clean / total_weight) * 2
    
%                 # Apply weights to losses
%                 per_sample_loss_weighted = per_sample_loss * weights
    
%                 # Compute mean loss and backpropagate
%                 loss = per_sample_loss_weighted.mean()
%                 loss.backward()
%                 optimizer.step()
    
%                 # Update running loss and total accuracy
%                 running_loss += loss.item() * batch_size
%                 correct_total += (predicted == labels).sum().item()
    
%                 digit_labels = labels // num_colors
%                 color_labels = labels % num_colors
%                 digit_predictions = predicted // num_colors
%                 color_predictions = predicted % num_colors
    
%                 correct_digit_total += (digit_predictions == digit_labels).sum().item()
%                 correct_color_total += (color_predictions == color_labels).sum().item()
    
%                 # Process noisy samples
%                 if num_noisy > 0:
%                     labels_noisy = labels[idx_noisy]
%                     predicted_noisy = predicted[idx_noisy]
%                     correct_noisy += (predicted_noisy == labels_noisy).sum().item()
%                     total_noisy += num_noisy
    
%                     # Update digit and color accuracy for noisy samples
%                     digit_labels_noisy = labels_noisy // num_colors
%                     color_labels_noisy = labels_noisy % num_colors
%                     digit_predictions_noisy = predicted_noisy // num_colors
%                     color_predictions_noisy = predicted_noisy % num_colors
    
%                     correct_digit_noisy += (digit_predictions_noisy == digit_labels_noisy).sum().item()
%                     correct_color_noisy += (color_predictions_noisy == color_labels_noisy).sum().item()
    
%                     # Store noisy sample losses
%                     loss_values_noisy.extend(per_sample_loss_weighted[idx_noisy].detach().cpu().numpy())
    
%                 # Process clean samples
%                 if num_clean > 0:
%                     labels_clean = labels[idx_clean]
%                     predicted_clean = predicted[idx_clean]
%                     correct_clean += (predicted_clean == labels_clean).sum().item()
%                     total_clean += num_clean
    
%                     # Update digit and color accuracy for clean samples
%                     digit_labels_clean = labels_clean // num_colors
%                     color_labels_clean = labels_clean % num_colors
%                     digit_predictions_clean = predicted_clean // num_colors
%                     color_predictions_clean = predicted_clean % num_colors
    
%                     correct_digit_clean += (digit_predictions_clean == digit_labels_clean).sum().item()
%                     correct_color_clean += (color_predictions_clean == color_labels_clean).sum().item()
    
%                     # Store clean sample losses
%                     loss_values_clean.extend(per_sample_loss_weighted[idx_clean].detach().cpu().numpy())
    
%                 # Store all losses
%                 loss_values.extend(per_sample_loss_weighted.detach().cpu().numpy())
    
%             except Exception as e:
%                 print(f"Error in training batch: {str(e)}")
%                 continue
    
%         # Calculate loss statistics
%         avg_loss = np.mean(loss_values) if loss_values else float('nan')
%         var_loss = np.var(loss_values) if loss_values else float('nan')
    
%         avg_loss_noisy = np.mean(loss_values_noisy) if loss_values_noisy else float('nan')
%         var_loss_noisy = np.var(loss_values_noisy) if loss_values_noisy else float('nan')
    
%         avg_loss_clean = np.mean(loss_values_clean) if loss_values_clean else float('nan')
%         var_loss_clean = np.var(loss_values_clean) if loss_values_clean else float('nan')
    
%         # Calculate accuracies
%         metrics = {
%             'avg_loss': avg_loss,
%             'var_loss': var_loss,
%             'accuracy_total': 100. * correct_total / total_samples if total_samples > 0 else float('nan'),
%             'accuracy_noisy': 100. * correct_noisy / total_noisy if total_noisy > 0 else float('nan'),
%             'accuracy_clean': 100. * correct_clean / total_clean if total_clean > 0 else float('nan'),
%             'avg_loss_noisy': avg_loss_noisy,
%             'var_loss_noisy': var_loss_noisy,
%             'avg_loss_clean': avg_loss_clean,
%             'var_loss_clean': var_loss_clean,
%             'accuracy_digit_total': 100. * correct_digit_total / total_samples if total_samples > 0 else float('nan'),
%             'accuracy_color_total': 100. * correct_color_total / total_samples if total_samples > 0 else float('nan'),
%             'accuracy_digit_noisy': 100. * correct_digit_noisy / total_noisy if total_noisy > 0 else float('nan'),
%             'accuracy_color_noisy': 100. * correct_color_noisy / total_noisy if total_noisy > 0 else float('nan'),
%             'accuracy_digit_clean': 100. * correct_digit_clean / total_clean if total_clean > 0 else float('nan'),
%             'accuracy_color_clean': 100. * correct_color_clean / total_clean if total_clean > 0 else float('nan'),
%             'total_samples': total_samples,
%             'total_noisy': total_noisy,
%             'total_clean': total_clean,
%             'correct_total': correct_total,
%             'correct_noisy': correct_noisy,
%             'correct_clean': correct_clean
%         }
    
%         return metrics
    
%     def test_model(model, test_loader, device, num_colors, num_digits):
%         model.eval()
%         loss_values = []  # 損失値を保存するリスト
%         test_loss = 0
%         correct_total = 0
%         total_samples = 0
    
%         correct_digit_total = 0
%         correct_color_total = 0
    
%         # criterion を 'none' に設定して個々のサンプルの損失を取得
%         criterion = nn.CrossEntropyLoss(reduction='none')
    
%         with torch.no_grad():
%             for inputs, labels in test_loader:
%                 inputs, labels = inputs.to(device), labels.to(device)
%                 outputs = model(inputs)
                
%                 # 個々のサンプルの損失を計算
%                 per_sample_loss = criterion(outputs, labels)
%                 loss_values.extend(per_sample_loss.cpu().numpy())
                
%                 # バッチ全体の損失を計算
%                 test_loss += per_sample_loss.mean().item() * labels.size(0)
                
%                 _, predicted = torch.max(outputs, 1)
%                 total_samples += labels.size(0)
%                 correct_total += (predicted == labels).sum().item()
    
%                 # Calculate correct counts for digits and colors
%                 digit_labels = labels // num_colors
%                 color_labels = labels % num_colors
%                 digit_predictions = predicted // num_colors
%                 color_predictions = predicted % num_colors
    
%                 correct_digit_total += (digit_predictions == digit_labels).sum().item()
%                 correct_color_total += (color_predictions == color_labels).sum().item()
    
%         # 損失の平均と分散を計算
%         avg_loss = test_loss / total_samples if total_samples > 0 else float('nan')
%         var_loss = np.var(loss_values) if loss_values else float('nan')
        
%         accuracy_total = 100. * correct_total / total_samples if total_samples > 0 else float('nan')
%         accuracy_digit_total = 100. * correct_digit_total / total_samples if total_samples > 0 else float('nan')
%         accuracy_color_total = 100. * correct_color_total / total_samples if total_samples > 0 else float('nan')
    
%         return {
%             'avg_loss': avg_loss,
%             'var_loss': var_loss,  # 追加
%             'accuracy_total': accuracy_total,
%             'accuracy_digit_total': accuracy_digit_total,
%             'accuracy_color_total': accuracy_color_total,
%             'total_samples': total_samples,
%             'correct_total': correct_total
%         }
        
%     def main():
%         """
%         Main training loop with comprehensive error handling and logging
%         """
%         print('Start session')
%         wandb_run = None
        
%         try:
%             # Parse arguments and set initial configurations
%             args = parse_args()
%             set_seed(args.fix_seed)
            
%             # Set device
%             device = set_device(args.gpu)
%             print(f'Using device: {device}')
    
%             # Load datasets with error handling
%             print('Loading datasets...')
%             try:
%                 train_dataset, test_dataset, imagesize, num_classes, in_channels = load_datasets(
%                     args.dataset, args.target, args.gray_scale, args)
%             except FileNotFoundError as e:
%                 print(f"Error loading dataset: {e}")
%                 return
%             except Exception as e:
%                 print(f"Unexpected error loading dataset: {e}")
%                 return
    
%             # Set number of digit and color classes
%             if args.target == 'combined':
%                 num_digits = 10
%                 num_colors = 10
%             else:
%                 num_digits = 10
%                 num_colors = 1
    
%             # Add label noise and create NoisyDataset
%             print(f'Preparing dataset with label noise rate: {args.label_noise_rate}')
%             if args.label_noise_rate > 0.0:
%                 if hasattr(train_dataset, 'tensors'):
%                     x_train, y_train = train_dataset.tensors
%                     y_train_noisy, noise_info = add_label_noise(y_train, args.label_noise_rate, num_digits, num_colors)
%                     train_dataset = torch.utils.data.TensorDataset(x_train, y_train_noisy)
%                     train_dataset = NoisyDataset(train_dataset, noise_info)
%                 else:
%                     y_train = torch.tensor(train_dataset.targets)
%                     y_train_noisy, noise_info = add_label_noise(y_train, args.label_noise_rate, num_digits, num_colors)
%                     train_dataset.targets = y_train_noisy.tolist()
%                     train_dataset = NoisyDataset(train_dataset, noise_info)
%             else:
%                 noise_info = torch.zeros(len(train_dataset), dtype=torch.int)
%                 train_dataset = NoisyDataset(train_dataset, noise_info)
    
%             # Extract indices for clean and noisy samples
%             clean_indices = [i for i, label in enumerate(noise_info) if label == 0]
%             noisy_indices = [i for i, label in enumerate(noise_info) if label == 1]
    
%             # Validate batch size
%             if args.batch_size % 2 != 0:
%                 raise ValueError("Batch size must be even for balanced batches")
    
%             # Create data loaders
%             print('Setting up data loaders...')
%             if args.label_noise_rate == 0.0 or args.label_noise_rate == 1.0:
%                 train_loader = torch.utils.data.DataLoader(
%                     train_dataset,
%                     batch_size=args.batch_size,
%                     shuffle=True,
%                     num_workers=args.num_workers,
%                     pin_memory=True
%                 )
%             else:
%                 batch_sampler = BalancedBatchSampler(
%                     clean_indices,
%                     noisy_indices,
%                     args.batch_size,
%                     drop_last=False
%                 )
%                 train_loader = torch.utils.data.DataLoader(
%                     train_dataset,
%                     batch_sampler=batch_sampler,
%                     num_workers=args.num_workers,
%                     pin_memory=True
%                 )
    
%             test_loader = torch.utils.data.DataLoader(
%                 test_dataset,
%                 batch_size=args.batch_size,
%                 shuffle=False,
%                 num_workers=args.num_workers,
%                 pin_memory=True
%             )
    
%             # Initialize model
%             print('Initializing model...')
%             model = load_models(in_channels, args, imagesize, num_classes)
%             model = model.to(device)
    
%             # Set optimizer
%             print('Setting up optimizer...')
%             if args.optimizer == "sgd":
%                 optimizer = optim.SGD(model.parameters(), lr=args.lr, momentum=args.momentum)
%             elif args.optimizer == "adam":
%                 optimizer = optim.Adam(model.parameters(), lr=args.lr)
%             elif args.optimizer == "adamw":
%                 optimizer = optim.AdamW(model.parameters(), lr=args.lr)
%             elif args.optimizer == "rmsprop":
%                 optimizer = optim.RMSprop(model.parameters(), lr=args.lr, momentum=args.momentum)
%             elif args.optimizer == "adagrad":
%                 optimizer = optim.Adagrad(model.parameters(), lr=args.lr)
%             else:
%                 raise ValueError(f"Unsupported optimizer: {args.optimizer}")
    
%             # Set loss function
%             criterion = nn.CrossEntropyLoss(reduction='none')
    
%             # Set experiment name
%             experiment_name = (
%                 f'{args.model}_{args.dataset}_{args.target}_'
%                 f'lr{args.lr}_batch{args.batch_size}_epoch{args.epoch}_'
%                 f'LabelNoiseRate{args.label_noise_rate}_Optim{args.optimizer}_'
%                 f'cleanw{args.weight_clean}_noisew{args.weight_noisy}_variance{args.variance}_width{args.model_width}_'
%                 f'seed{args.fix_seed}'
                
%             )
%             print(f'Experiment name: {experiment_name}')
    
%             # Initialize wandb
%             if args.wandb:
%                 print('Initializing wandb...')
%                 wandb_run = wandb.init(
%                     project=args.wandb_project,
%                     name=experiment_name,
%                     entity=args.wandb_entity,
%                     config=args
%                 )
    
%             # Set up CSV logging
%             csv_dir = f"./csv/combine/split_noise/{experiment_name}"
%             os.makedirs(csv_dir, exist_ok=True)
%             csv_path = os.path.join(csv_dir, 'log.csv')
            
%             if not os.path.isfile(csv_path):
%                 with open(csv_path, 'w', newline='') as f:
%                     writer = csv.writer(f)
%                     writer.writerow([
%                         "epoch", 
%                         "train_loss", "train_loss_variance",
%                         "train_accuracy", "train_accuracy_noisy", "train_accuracy_clean",
%                         "train_accuracy_digit_total", "train_accuracy_color_total",
%                         "train_accuracy_digit_noisy", "train_accuracy_color_noisy",
%                         "train_accuracy_digit_clean", "train_accuracy_color_clean",
%                         "test_loss", "test_loss_variance",
%                         "test_accuracy",
%                         "test_accuracy_digit_total", "test_accuracy_color_total"
%                     ])
    
%             # Training loop
%             print('Starting training...')
%             best_test_accuracy = 0.0
%             for epoch in range(1, args.epoch + 1):
%                 try:
%                     # Training phase
%                     train_metrics = train_model(
%                         model=model,
%                         train_loader=train_loader,
%                         optimizer=optimizer,
%                         criterion=criterion,
%                         weight_noisy=args.weight_noisy,
%                         weight_clean=args.weight_clean,
%                         device=device,
%                         num_colors=num_colors,
%                         num_digits=num_digits
%                     )
    
%                     # Testing phase
%                     test_metrics = test_model(
%                         model=model,
%                         test_loader=test_loader,
%                         device=device,
%                         num_colors=num_colors,
%                         num_digits=num_digits
%                     )
    
%                     # Print progress
%                     print(f"\nEpoch: {epoch}/{args.epoch}")
%                     print(f"Train Loss: {train_metrics['avg_loss']:.4f}, "
%                           f"Train Loss Variance: {train_metrics['var_loss']:.4f}")
%                     print(f"Train Accuracy: {train_metrics['accuracy_total']:.2f}%, "
%                           f"Test Accuracy: {test_metrics['accuracy_total']:.2f}%")
%                     print(f"Train Digit/Color Accuracy: {train_metrics['accuracy_digit_total']:.2f}%/"
%                           f"{train_metrics['accuracy_color_total']:.2f}%")
%                     print(f"Test Digit/Color Accuracy: {test_metrics['accuracy_digit_total']:.2f}%/"
%                           f"{test_metrics['accuracy_color_total']:.2f}%")
    
%                     # Save to CSV
%                     with open(csv_path, 'a', newline='') as f:
%                         writer = csv.writer(f)
%                         writer.writerow([
%                             epoch,
%                             train_metrics['avg_loss'], train_metrics['var_loss'],
%                             train_metrics['accuracy_total'],
%                             train_metrics['accuracy_noisy'],
%                             train_metrics['accuracy_clean'],
%                             train_metrics['accuracy_digit_total'],
%                             train_metrics['accuracy_color_total'],
%                             train_metrics['accuracy_digit_noisy'],
%                             train_metrics['accuracy_color_noisy'],
%                             train_metrics['accuracy_digit_clean'],
%                             train_metrics['accuracy_color_clean'],
%                             test_metrics['avg_loss'], test_metrics['var_loss'],
%                             test_metrics['accuracy_total'],
%                             test_metrics['accuracy_digit_total'],
%                             test_metrics['accuracy_color_total']
%                         ])
    
%                     # Log to wandb
%                     if args.wandb:
%                         wandb.log({
%                             'epoch': epoch,
%                             'train_loss': train_metrics['avg_loss'],
%                             'train_loss_variance': train_metrics['var_loss'],
%                             'train_accuracy': train_metrics['accuracy_total'],
%                             'train_accuracy_noisy': train_metrics['accuracy_noisy'],
%                             'train_accuracy_clean': train_metrics['accuracy_clean'],
%                             'train_accuracy_digit_total': train_metrics['accuracy_digit_total'],
%                             'train_accuracy_color_total': train_metrics['accuracy_color_total'],
%                             'train_accuracy_digit_noisy': train_metrics['accuracy_digit_noisy'],
%                             'train_accuracy_color_noisy': train_metrics['accuracy_color_noisy'],
%                             'train_accuracy_digit_clean': train_metrics['accuracy_digit_clean'],
%                             'train_accuracy_color_clean': train_metrics['accuracy_color_clean'],
%                             'test_loss': test_metrics['avg_loss'],
%                             'test_loss_variance': test_metrics['var_loss'],
%                             'test_accuracy': test_metrics['accuracy_total'],
%                             'test_accuracy_digit_total': test_metrics['accuracy_digit_total'],
%                             'test_accuracy_color_total': test_metrics['accuracy_color_total']
%                         })
    
%                     # Memory management
%                     if epoch % 10 == 0:
%                         clear_memory()
    
%                     # Save best model
%                     if test_metrics['accuracy_total'] > best_test_accuracy:
%                         best_test_accuracy = test_metrics['accuracy_total']
%                         torch.save(model.state_dict(), 
%                                  os.path.join(csv_dir, 'best_model.pth'))
    
%                 except Exception as e:
%                     print(f"Error in epoch {epoch}: {str(e)}")
%                     continue
    
%         except Exception as e:
%             print(f"Fatal error in training: {str(e)}")
%             raise
        
%         finally:
%             # Cleanup
%             if wandb_run is not None:
%                 wandb_run.finish()
%             print('Training completed')
    
%     if __name__ == '__main__':
%         main()
% \end{lstlisting}


% \chapter{対外成果発表リスト}
\begin{enumerate}
    \item[2022-03-15] \underline{\UTF{9AD9}橋 秀弥}, 鏡川 悠介,前田 英作, ``深層ニューラルネットワークにおける二重降下現象
,''  電子情報通信学会2022年総合大会 情報・システムソサイエティ特別企画 ジュニア&学生ポスターセッション予稿集, ISS-SP-028, ポスター発表, 開催地 zoom.
    \item[2022-04-16] 杉田 拓磨, 岡澤 律来, 金田 龍平, \underline{\UTF{9AD9}橋 秀弥}, 鏡川 悠介, 出場チーム名 AID, AIイノベーションアワード2022, 開催地 立教大学, \\最優秀賞(https://www.nttpc.co.jp/press/2022/04/202204211500.html).
    \item[2022-07-26] \underline{\UTF{9AD9}橋 秀弥}, 鏡川 悠介,前田 英作, ``深層学習における二重降下現象と画像のテクスチャ・形状性について
,''  第25回 画像の認識・理解シンポジウム(MIRU2022), OS1A-5, 口頭発表(査読あり),ポスター発表, 開催地 姫路市文化コンベンションセンター  アクリエひめじ(兵庫県).
    \item[2022-07-28] 杉田 拓磨, 岡澤 律来, 金田 龍平, \underline{\UTF{9AD9}橋 秀弥}, 鏡川 悠介, 前田 英作, ``物語文を入力とする自動挿絵生成システム
,''  第25回 画像の認識・理解シンポジウム(MIRU2022), IS3-75, ポスター発表, 開催地 姫路市文化コンベンションセンター  アクリエひめじ(兵庫県).
    \item[2023-03-02] \underline{\UTF{9AD9}橋 秀弥},井上中順,横田理央,片岡裕雄,前田英作, ``画像識別における形状・テクスチャ偏重度と二重降下現象の関係について
,''  パターン認識・メディア理解(PRMU)2023年3月研究会, PRMU-3, 口頭発表(査読なし), 公立はこだて未来大(北海道).
    \item[2023-03-02] 遠藤隆斗,\underline{\UTF{9AD9}橋 秀弥},前田英作, ``医療画像タスクにおける数式駆動型教師あり学習の有効性について
,''  パターン認識・メディア理解(PRMU)2023年3月研究会, PRMU-22, 口頭発表(査読なし), 公立はこだて未来大(北海道).
    \item[2023-07-28] \underline{\UTF{9AD9}橋 秀弥},井上中順,横田理央,片岡裕雄,前田英作, ``学習過程における形状・テクスチャ偏重度の推移と事前学習データセットとの関係について
,''  第26回 画像の認識・理解シンポジウム(MIRU2023), IS3-26,ポスター発表, 開催地 アクトシティ浜松(静岡県).




\end{enumerate}


\bibliographystyle{IEEEtran}
\bibliography{ref/mybibfile.bib}

\end{document}