\chapter{結論}
本研究では,深層ニューラルネットワーク(DNN)の学習過程における視覚的概念の獲得メカニズムについて,
数字と色という2つの明確に定義された概念を用いて実験的な検討を行った.
研究の成果として,まずEMNISTデータセットに色情報を付加した新しいデータセットを設計し,
概念獲得過程を定量的に評価できる実験環境を確立した.
この環境では,色のノイズレベルとラベルノイズを独立に制御することが可能であり,各概念の難易度を系統的に調整できることを示した.
実験の結果,色と数字の概念獲得には明確な順序性が存在し,より単純な概念である色が先に学習される傾向が確認された.
また,モデルの層の深さやチャネル幅によって,概念獲得の順序や相互作用が変化することが明らかとなった.
特に,チャネル幅が広いモデルでは並列的な概念学習が可能となることを示した.
さらに,二重降下現象と概念獲得の関係性を明らかにし,学習の各フェーズにおける特徴表現の変化を解明した.
加えて,ノイズに対する頑健性がモデルのアーキテクチャに依存することを示し,適切なモデル設計の指針を提供した.
これらの知見は,DNNの学習メカニズムの理解を深めるとともに,
より効率的で解釈可能な深層学習モデルの設計に向けた重要な示唆を与えるものである.
