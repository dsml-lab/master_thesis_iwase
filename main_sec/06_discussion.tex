\chapter{考察}
まず,5層のCNNと8層のCNNの実験結果から,アーキテクチャの深さが概念獲得プロセスに本質的な影響を与えることが明らかとなった.
5層のCNNは8層のCNNと比較して,Phase1における色と数字の並列学習をより効率的に実現している.
この結果は,ネットワークの深さが増すことで,異なる視覚的特徴を同時に学習する能力が向上するが,
その学習が始まるためにより多くのニューロンが活性化しなければならないことを示唆している.
性能に関して,より深い層構造により,特徴抽出の階層性が高まり,色情報と形状情報を独立して処理できる表現空間が獲得されたためと考えられる.
特に興味深い点として,8層のCNNでは色と数字の認識精度が同期的に向上する傾向が観測された.
これは,より深い層構造が特徴の階層的な抽象化を促進し,複数の視覚的概念を効率的に並列処理できる表現を獲得できることを示している.
また,学習過程における二重降下現象の出現パターンがモデルの深さによって質的に異なることが観測された.
特に,8層のCNNはPhase2からPhase3への移行期において,CleanとNoisyの学習曲線が5層のCNNより早期に収束する傾向を示した.
この結果は,より深いアーキテクチャが,ノイズの影響を効果的に抑制しながら本質的な特徴を学習できることを示している.
さらに,この現象は従来の二重降下現象の研究では十分に説明されていなかった,アーキテクチャの深さと学習ダイナミクスの関係性に新たな知見を提供するものである.
色認識と数字認識の学習タイミングに関して,両モデルで顕著に異なる特性が観測された.
5層のCNNがPhase1において訓練データに対する色認識が50\%の正解率で停滞し始めるタイミングがより早い.
この結果は,ネットワークの表現能力が,概念獲得の時間的順序影響を与えることを示唆している.
特筆すべき点として,8層のCNNではPhase2における上昇が抑えられて,進行することが確認された.
これは,より深い層構造が,異なる視覚的特徴間の関係性を効果的に捉える能力を持つことを示していると考える.
次に,RGBノイズの分散を変化させた実験について考察する.$\sigma = 10\sqrt{10}$,$\sigma = 100$のノイズ条件下において,
色に対する難易度を上げても,ほとんど数字に対する誤り率の変化は見られなかった.
これは,CNNのモデル内部において,数字概念と色概念を分けて学習して,概念を獲得していることを示唆している.
本研究で観測された学習ダイナミクスは,従来のshape/texture biasの研究では捉えきれなかった重要な側面を明らかにした.
特に,明確に定義された概念(数字と色)を用いた分析アプローチにより,CNNの学習過程における概念獲得のメカニズムをより精緻に理解することが可能となった.
この知見は,今後の深層学習グモデルの設計において,層の深さと概念獲得能力の関係性を考慮する重要性を示唆している.
今後の研究課題として,より多様な視覚的概念や,異なる層構造を持つアーキテクチャでの検証が挙げられる.
また,本研究で観測された学習ダイナミクスの理論的な解明も重要な課題である.
これらの課題に取り組むことで,CNNの学習メカニズムのより深い理解と,より高度な視覚認識システムの実現が期待される.
