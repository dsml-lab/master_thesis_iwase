\chapter{結論}
本稿では,epoch-wise double descentの先行研究に触発され,まだ理論的な解析がなされていない画像固有の特徴(形状・テクスチャ)と二重降下の関係に着目した.まず,画像特徴の獲得過程を追跡するため,既存の手法を用いて形状・テクスチャに対するモデルの偏重度を定量化し,この偏重度と学習中のテスト誤差の推移を比較した.結果として,ImageNetを事前学習した場合に,いくつかの条件下では,学習過程における形状・テクスチャ偏重度の推移とテスト誤り率が描くepoch-wise double descentの推移に相関がみられることがわかった.また,定量的な評価を通して,テスト誤り率の一度目の下降から上昇しきるまでの区間において特に相関が確認された.

さらに,定性的な可視化により,初期層におけるフィルタ,レイヤーの深さに基づく形状・テクスチャ偏重度の変化を示した.より深い層においては形状・テクスチャ偏重度に変化を示すが,初期層のフィルターはほとんど変化しないことが観察された.このような結果から,double descentの観点においては,深い層に着目するべきであると考えられる.我々の研究は,epoch-wise double descentと,深層学習と二重降下の一般的な分野の両方について,より広い理解に貢献すると考える.
\newpage