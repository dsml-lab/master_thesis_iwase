% \chapter*{付録}
% \addcontentsline{toc}{chapter}{付録}
\chapter{学習プログラム}
\label{学習プログラム}
学習に利用したプログラムを示す.掲載したプログラムは,GitHub上で公開している.
\\
\url{https://github.com/dsml-lab/double_descent_and_shape_texture_bias}
% カスタムスタイルの定義
\lstdefinestyle{pythonstyle}{
    language=Python,
    basicstyle=\ttfamily\small,
    keywordstyle=\color{blue},
    stringstyle=\color{orange},
    commentstyle=\color{gray}\itshape,
    numbers=left,
    numberstyle=\tiny\color{gray},
    stepnumber=1,
    numbersep=5pt,
    backgroundcolor=\color{lightgray!10},
    frame=single,
    breaklines=true,
    showstringspaces=false,
    captionpos=b
}

% \begin{lstlisting}[style=pythonstyle, caption={メインコード}]
%     # combineのラベルノイズに関して、色・数字のラベル両方異なるラベルにする
%     # accuracyに関して、combineの正解率だけでなく、色・数字の正解率も出力する
%     # ラベルノイズの正解率とラベルノイズでない正解率を出力する
%     # 平均・分散も出力する
%     # 重みの加え方が均等
    
%     import os
%     import torch
%     import torch.nn as nn
%     import torch.optim as optim
%     import torch.nn.functional as f
%     import torch.backends.cudnn as cudnn
%     from torch.cuda.amp import autocast, GradScaler
%     from torch.utils.data.sampler import Sampler
%     import torchvision.transforms as transforms
%     from torch.utils.data import DataLoader, TensorDataset
%     from sklearn.metrics import classification_report
%     from sklearn.metrics import accuracy_score
%     import math
%     import torchvision.models as models
%     import numpy as np
%     import time
%     import wandb
%     import argparse
%     import random
%     import matplotlib.pyplot as plt
%     import csv 
%     import warnings
%     import gc
%     import gzip
%     # import models written by scratch
%     from model.cnn_2layers import CNN2Layer
%     from model.cnn_3layers import CNN3Layer
%     from model.cnn_4layers import CNN4Layer
%     from model.cnn_5layers import CNN5Layer
%     from model.cnn_8layers import CNN8Layer
%     from model.cnn_16layers import CNN16Layer
%     from model.resnet18 import ResNet18
%     # download datasets from pytorch
%     from torchvision import datasets
    
%     # Ignore warnings
%     warnings.filterwarnings("ignore")
%     # os.environ['CUDA_LAUNCH_BLOCKING'] = "1"
%     # os.environ['PYTORCH_CUDA_ALLOC_CONF'] = 'max_split_size_mb:128'
%     torch.backends.cudnn.benchmark = True
    
%     # settings
%     def parse_args():
%         """
%         Parse the command-line arguments.
        
%         Returns:
%             argparse.Namespace: The parsed command-line arguments.
%         """
%         arg_parser = argparse.ArgumentParser()
%         #set seed
%         arg_parser.add_argument("-seed", "--fix_seed", type=int, default=42)
        
%         #set model settings
%         arg_parser.add_argument("--model", type=str, choices=["cnn_2layers", "cnn_3layers", "cnn_4layers", "cnn_5layers", "cnn_8layers", "cnn_16layers", "resnet18"], help="モデルアーキテクチャの選択")
%         arg_parser.add_argument("-model_width", "--model_width", type=int, default=1)
%         arg_parser.add_argument("-epoch", "--epoch", type=int, default=1000)
        
%         #set dataset setting
%         arg_parser.add_argument("-datasets", "--dataset", type=str, choices=["mnist", "emnist", "emnist_digits", "cifar10", "cifar100", "tinyImageNet", "colored_emnist", "distribution_colored_emnist"], default="cifar10")
%         arg_parser.add_argument("-variance", "--variance", type=int, default=10000)
%         arg_parser.add_argument("-correlation", "--correlation", type=float, default=0.5)
%         arg_parser.add_argument("-label_noise_rate", "--label_noise_rate", type=float, default=0.0) 
%         arg_parser.add_argument("-gray_scale", "--gray_scale", action='store_true', help="グレースケールに変換するかどうか")
%         arg_parser.add_argument("-batch_size", "--batch_size", type=int, default=128, help="バッチサイズ")
%         arg_parser.add_argument("-img_size", "--img_size", type=int, default=32, help="画像サイズ")
%         arg_parser.add_argument("-target", "--target", type=str, choices=["color", "digit", "combined"], default='color', help="colored EMNISTのターゲットの指定:color or digit or combined")
        
%         # set optimizer setting
%         arg_parser.add_argument("-lr", "--lr", type=float, default=0.1, help="学習率")
%         arg_parser.add_argument("-optimizer", "--optimizer", type=str, choices=["sgd", "adam", "adamw", "rmsprop", "adagrad"], default="adam", help="最適化手法.adam were used in Nakkiran et al. (2019)")
%         arg_parser.add_argument("-momentum", "--momentum", type=float, default=0.9, help="モーメンタム")
        
%         #set loss function setting
%         arg_parser.add_argument("-loss", "--loss", type=str, choices=["cross_entropy", "focal_loss"], default="cross_entropy", help="損失関数")
        
%         # set device setting
%         arg_parser.add_argument("-gpu", "--gpu", type=int, default=0, help="GPU device ID")
%         arg_parser.add_argument("-num_workers", "--num_workers", type=int, default=4, help="データローダーの並列数")
        
%         # wandb setting
%         arg_parser.add_argument("-wandb", "--wandb", action='store_true',default=True ,help="wandbを使用するかどうか")
%         arg_parser.add_argument("-wandb_project", "--wandb_project", type=str, default="dd_scratch_models", help="wandbのプロジェクト名")
%         arg_parser.add_argument("--wandb_entity", type=str, default="dsml-kernel24", help="wandbのエンティティ名")
        
        
%         arg_parser.add_argument("-weight_noisy", "--weight_noisy", type=float, default=1.0, help="Weight for the loss of noisy samples")
%         arg_parser.add_argument("-weight_clean", "--weight_clean", type=float, default=1.0, help="Weight for the loss of clean samples")
%         return arg_parser.parse_args()
    
%     # Define a custom dataset that includes noise_info
%     class NoisyDataset(torch.utils.data.Dataset):
%         def __init__(self, dataset, noise_info):
%             self.dataset = dataset
%             self.noise_info = noise_info
    
%         def __len__(self):
%             return len(self.dataset)
    
%         def __getitem__(self, idx):
%             input, label = self.dataset[idx]
%             noise_label = self.noise_info[idx]
%             return input, label, noise_label
    
%     # Custom sampler to create balanced batches
%     class BalancedBatchSampler(Sampler):
%         def __init__(self, clean_indices, noisy_indices, batch_size, drop_last):
%             self.clean_indices = clean_indices
%             self.noisy_indices = noisy_indices
%             self.batch_size = batch_size
%             self.drop_last = drop_last
    
%             assert batch_size % 2 == 0, "Batch size must be even for balanced batches"
%             self.num_samples_per_class = batch_size // 2
    
%         def __iter__(self):
%             # Shuffle the indices
%             random.shuffle(self.clean_indices)
%             random.shuffle(self.noisy_indices)
    
%             # Calculate the number of batches
%             min_len = min(len(self.clean_indices), len(self.noisy_indices))
%             num_batches = min_len // self.num_samples_per_class
    
%             for i in range(num_batches):
%                 clean_batch = self.clean_indices[i * self.num_samples_per_class: (i + 1) * self.num_samples_per_class]
%                 noisy_batch = self.noisy_indices[i * self.num_samples_per_class: (i + 1) * self.num_samples_per_class]
%                 batch = clean_batch + noisy_batch
%                 random.shuffle(batch)
%                 yield batch
    
%             if not self.drop_last:
%                 # Handle remaining samples
%                 remaining_clean = self.clean_indices[num_batches * self.num_samples_per_class:]
%                 remaining_noisy = self.noisy_indices[num_batches * self.num_samples_per_class:]
    
%                 if len(remaining_clean) >= self.num_samples_per_class and len(remaining_noisy) >= self.num_samples_per_class:
%                     batch = remaining_clean[:self.num_samples_per_class] + remaining_noisy[:self.num_samples_per_class]
%                     random.shuffle(batch)
%                     yield batch
    
%         def __len__(self):
%             return len(self.clean_indices) // self.num_samples_per_class
    
%     # Set seeds
%     def set_seed(seed):
%         """
%         Set the seed for reproducibility.
    
%         Args:
%             seed (int): The seed value to set.
    
%         Returns:
%             None
%         """
%         random.seed(seed)
%         np.random.seed(seed)
%         torch.manual_seed(seed)
%         torch.cuda.manual_seed(seed)
%         torch.cuda.manual_seed_all(seed)
%         # For GPU determinism
%         torch.backends.cudnn.deterministic = True
%         torch.backends.cudnn.benchmark = False
    
%     # Set device
%     def set_device(gpu_id):
%         """
%         Sets the device for computation.
    
%         Args:
%             gpu_id (int): The ID of the GPU to use.
    
%         Returns:
%             torch.device: The selected device (GPU or CPU).
%         """
%         # Choose the GPU device if available, otherwise use CPU
%         device = torch.device("cuda:{}".format(gpu_id) if torch.cuda.is_available() else "cpu")
%         return device
    
%     def clear_memory():
%         torch.cuda.empty_cache()  # Clear CUDA cache
%         gc.collect()  # Force garbage collection
    
%     def apply_transform(x, transform):
%         transformed_x = []
%         for img in x:
%             img = transform(img)
%             transformed_x.append(img)
%         return torch.stack(transformed_x)
    
%     def load_datasets(dataset, target, gray_scale, args):
%         """
%         Load the specified dataset and apply transformations based on the dataset type and grayscale option.
        
%         Args:
%             dataset (str): The name of the dataset to load. Supported options are "mnist", "emnist", "cifar10", "cifar100", and "tinyImageNet".
%             gray_scale (bool): Flag indicating whether to convert the images to grayscale.
            
%         Returns:
%             tuple: A tuple containing the train dataset, test dataset, image size, and number of classes.
%         """
%         if dataset == "mnist":
%             transform = transforms.Compose([
%                 transforms.ToTensor(),
%                 transforms.Resize((32, 32)),
%                 transforms.Normalize((0.1307,), (0.3081,))
%             ])
%             train_dataset = datasets.MNIST(root='./data', train=True, download=True, transform=transform)
%             test_dataset = datasets.MNIST(root='./data', train=False, download=True, transform=transform)
%             imagesize = (32, 32)
%             num_classes = 10
%             in_channels = 1
%         elif dataset == "emnist":
%             transform = transforms.Compose([
%                 transforms.ToTensor(),
%                 transforms.Resize((32, 32)),
%                 transforms.Normalize((0.1307,), (0.3081,))
%             ])
%             train_dataset = datasets.EMNIST(root='./data', split='balanced', train=True, download=True, transform=transform)
%             test_dataset = datasets.EMNIST(root='./data', split='balanced', train=False, download=True, transform=transform)
%             imagesize = (32, 32)
%             num_classes = 47
%             in_channels = 1
%         elif dataset == "emnist_digits":
%             emnist_path = './data/EMNIST'
%             def load_gz_file(file_path, is_image=True):
%                 with gzip.open(file_path, 'rb') as f:
%                     if is_image:
%                         return np.frombuffer(f.read(), dtype=np.uint8, offset=16).reshape(-1, 28, 28)
%                     else:
%                         return np.frombuffer(f.read(), dtype=np.uint8, offset=8)
    
%             x_train = load_gz_file(os.path.join(emnist_path, 'emnist-digits-train-images-idx3-ubyte.gz'))
%             y_train = load_gz_file(os.path.join(emnist_path, 'emnist-digits-train-labels-idx1-ubyte.gz'), is_image=False)
%             x_test = load_gz_file(os.path.join(emnist_path, 'emnist-digits-test-images-idx3-ubyte.gz'))
%             y_test = load_gz_file(os.path.join(emnist_path, 'emnist-digits-test-labels-idx1-ubyte.gz'), is_image=False)
%             # 変換関数が必要な場合はここで定義
%             transform = transforms.Compose([
%                 transforms.ToPILImage(),  # Convert numpy array to PIL Image
%                 transforms.Resize((32, 32)),  # Same size as original, adjust if needed
%                 transforms.ToTensor()
%             ])
    
%             # Apply transformation
%             x_train_tensor = apply_transform(x_train, transform)
%             x_test_tensor = apply_transform(x_test, transform)
    
%             y_train_tensor = torch.tensor(y_train, dtype=torch.long)
%             y_test_tensor = torch.tensor(y_test, dtype=torch.long)
    
%             train_dataset = torch.utils.data.TensorDataset(x_train_tensor, y_train_tensor)
%             test_dataset = torch.utils.data.TensorDataset(x_test_tensor, y_test_tensor)
    
%             num_classes = 10  # Digits from 0 to 9
%             in_channels = 1  # Grayscale images
%             imagesize = (32, 32)  # Original image size
%         elif dataset == "colored_emnist":
%             # target: color or digit or combined
            
%             # Data augmentation
%             transform = transforms.Compose([
%                 transforms.ToPILImage(),  # Convert numpy array to PIL Image
%                 transforms.Resize((32, 32)),
%                 transforms.ToTensor(),
%                 transforms.Normalize((0.1307,), (0.3081,))
%             ])
            
%             if target == 'color':
%                 x_train = np.load('data/colored_EMNIST/x_train_colored.npy')
%                 y_train_colors = np.load('data/colored_EMNIST/y_train_colors.npy')
%                 x_test = np.load('data/colored_EMNIST/x_test_colored.npy')
%                 y_test_colors = np.load('data/colored_EMNIST/y_test_colors.npy')
                
%                 # Apply transformation
%                 x_train_tensor = apply_transform(x_train, transform)
%                 x_test_tensor = apply_transform(x_test, transform)
                
%                 y_train_tensor = torch.tensor(y_train_colors, dtype=torch.long)
%                 y_test_tensor = torch.tensor(y_test_colors, dtype=torch.long)
                
%                 train_dataset = torch.utils.data.TensorDataset(x_train_tensor, y_train_tensor)
%                 test_dataset = torch.utils.data.TensorDataset(x_test_tensor, y_test_tensor)
            
%             elif target == 'digit':
%                 x_train = np.load('data/colored_EMNIST/x_train_colored.npy')
%                 y_train_digits = np.load('data/colored_EMNIST/y_train_digits.npy')
%                 x_test = np.load('data/colored_EMNIST/x_test_colored.npy')
%                 y_test_digits = np.load('data/colored_EMNIST/y_test_digits.npy')
                
%                 # Apply transformation
%                 x_train_tensor = apply_transform(x_train, transform)
%                 x_test_tensor = apply_transform(x_test, transform)
                
%                 y_train_tensor = torch.tensor(y_train_digits, dtype=torch.long)
%                 y_test_tensor = torch.tensor(y_test_digits, dtype=torch.long)
                
%                 train_dataset = torch.utils.data.TensorDataset(x_train_tensor, y_train_tensor)
%                 test_dataset = torch.utils.data.TensorDataset(x_test_tensor, y_test_tensor)
                
%             elif target == 'combined':
%                 x_train = np.load('data/colored_EMNIST/x_train_colored.npy')
%                 y_train_combined = np.load('data/colored_EMNIST/y_train_combined.npy')
%                 x_test = np.load('data/colored_EMNIST/x_test_colored.npy')
%                 y_test_combined = np.load('data/colored_EMNIST/y_test_combined.npy')
                
%                 # Apply transformation
%                 x_train_tensor = apply_transform(x_train, transform)
%                 x_test_tensor = apply_transform(x_test, transform)
                
%                 y_train_tensor = torch.tensor(y_train_combined, dtype=torch.long)
%                 y_test_tensor = torch.tensor(y_test_combined, dtype=torch.long)
                
%                 train_dataset = torch.utils.data.TensorDataset(x_train_tensor, y_train_tensor)
%                 test_dataset = torch.utils.data.TensorDataset(x_test_tensor, y_test_tensor)
                
%             num_classes = 10 if target in ['color', 'digit'] else 100
%             in_channels = 3
%             imagesize = (32, 32)
%         elif dataset == "distribution_colored_emnist":
%             # target: color or digit or combined
%             seed = args.fix_seed
%             variance = args.variance
%             correlation = args.correlation
%             # Data augmentation
%             transform = transforms.Compose([
%                 transforms.ToPILImage(),  # Convert numpy array to PIL Image
%                 transforms.Resize((32, 32)),
%                 transforms.ToTensor(),
%                 transforms.Normalize((0.1307,), (0.3081,))
%             ])
            
%             if target == 'color':
%                 x_train = np.load(f'data/distribution_colored_EMNIST_Seed42_Var{variance}_Corr{correlation}/x_train_colored.npy')
%                 y_train_colors = np.load(f'data/distribution_colored_EMNIST_Seed42_Var{variance}_Corr{correlation}/y_train_colors.npy')
%                 x_test = np.load(f'data/distribution_colored_EMNIST_Seed42_Var{variance}_Corr{correlation}/x_test_colored.npy')
%                 y_test_colors = np.load(f'data/distribution_colored_EMNIST_Seed42_Var{variance}_Corr{correlation}/y_test_colors.npy')
                
%                 # Apply transformation
%                 x_train_tensor = apply_transform(x_train, transform)
%                 x_test_tensor = apply_transform(x_test, transform)
                
%                 y_train_tensor = torch.tensor(y_train_colors, dtype=torch.long)
%                 y_test_tensor = torch.tensor(y_test_colors, dtype=torch.long)
                
%                 train_dataset = torch.utils.data.TensorDataset(x_train_tensor, y_train_tensor)
%                 test_dataset = torch.utils.data.TensorDataset(x_test_tensor, y_test_tensor)
            
%             elif target == 'digit':
%                 x_train = np.load(f'data/distribution_colored_EMNIST_Seed42_Var{variance}_Corr{correlation}/x_train_colored.npy')
%                 y_train_digits = np.load(f'data/distribution_colored_EMNIST_Seed42_Var{variance}_Corr{correlation}/y_train_digits.npy')
%                 x_test = np.load(f'data/distribution_colored_EMNIST_Seed42_Var{variance}_Corr{correlation}/x_test_colored.npy')
%                 y_test_digits = np.load(f'data/distribution_colored_EMNIST_Seed42_Var{variance}_Corr{correlation}/y_test_digits.npy')
                
%                 # Apply transformation
%                 x_train_tensor = apply_transform(x_train, transform)
%                 x_test_tensor = apply_transform(x_test, transform)
                
%                 y_train_tensor = torch.tensor(y_train_digits, dtype=torch.long)
%                 y_test_tensor = torch.tensor(y_test_digits, dtype=torch.long)
                
%                 train_dataset = torch.utils.data.TensorDataset(x_train_tensor, y_train_tensor)
%                 test_dataset = torch.utils.data.TensorDataset(x_test_tensor, y_test_tensor)
                
%             elif target == 'combined':
%                 x_train = np.load(f'data/distribution_colored_EMNIST_Seed42_Var{variance}_Corr{correlation}/x_train_colored.npy')
%                 y_train_combined = np.load(f'data/distribution_colored_EMNIST_Seed42_Var{variance}_Corr{correlation}/y_train_combined.npy')
%                 x_test = np.load(f'data/distribution_colored_EMNIST_Seed42_Var{variance}_Corr{correlation}/x_test_colored.npy')
%                 y_test_combined = np.load(f'data/distribution_colored_EMNIST_Seed42_Var{variance}_Corr{correlation}/y_test_combined.npy')
                
%                 # Apply transformation
%                 x_train_tensor = apply_transform(x_train, transform)
%                 x_test_tensor = apply_transform(x_test, transform)
                
%                 y_train_tensor = torch.tensor(y_train_combined, dtype=torch.long)
%                 y_test_tensor = torch.tensor(y_test_combined, dtype=torch.long)
                
%                 train_dataset = torch.utils.data.TensorDataset(x_train_tensor, y_train_tensor)
%                 test_dataset = torch.utils.data.TensorDataset(x_test_tensor, y_test_tensor)
        
%             num_classes = 10 if target in ['color', 'digit'] else 100
%             in_channels = 3
%             imagesize = (32, 32)
%         elif dataset == "cifar10":
%             transform = transforms.Compose([
%                 transforms.RandomCrop(32, padding=4),
%                 transforms.RandomHorizontalFlip(),
%                 transforms.ToTensor(),
%                 transforms.Normalize((0.4914, 0.4822, 0.4465), (0.2023, 0.1994, 0.2010))
%             ])
%             train_dataset = datasets.CIFAR10(root='./data', train=True, download=True, transform=transform)
%             test_dataset = datasets.CIFAR10(root='./data', train=False, download=True, transform=transform)
%             imagesize = (32, 32)
%             num_classes = 10
%             in_channels = 3
%         elif dataset == "cifar100":
%             transform = transforms.Compose([
%                 transforms.RandomCrop(32, padding=4),
%                 transforms.RandomHorizontalFlip(),
%                 transforms.ToTensor(),
%                 transforms.Normalize((0.5071, 0.4867, 0.4408), (0.2675, 0.2565, 0.2761))
%             ])
%             train_dataset = datasets.CIFAR100(root='./data', train=True, download=True, transform=transform)
%             test_dataset = datasets.CIFAR100(root='./data', train=False, download=True, transform=transform)
%             imagesize = (32, 32)
%             num_classes = 100
%             in_channels = 3
%         elif dataset == "tinyImageNet":
%             transform = transforms.Compose([
%                 transforms.RandomCrop(64, padding=4),
%                 transforms.RandomHorizontalFlip(),
%                 transforms.ToTensor(),
%                 transforms.Normalize((0.4802, 0.4481, 0.3975), (0.2302, 0.2265, 0.2262))
%             ])
%             train_dataset = datasets.ImageFolder(root='./data/tiny-imagenet-200/train', transform=transform)
%             test_dataset = datasets.ImageFolder(root='./data/tiny-imagenet-200/val', transform=transform)
%             imagesize = (64, 64)
%             num_classes = 200
%             in_channels = 3
%         elif dataset == "distribution_to_normal":
%             # target: color or digit or combined
%             seed = args.fix_seed
%             variance = args.variance
%             correlation = args.correlation
%             # Data augmentation
%             transform = transforms.Compose([
%                 transforms.ToPILImage(),  # Convert numpy array to PIL Image
%                 transforms.Resize((32, 32)),
%                 transforms.ToTensor(),
%                 transforms.Normalize((0.1307,), (0.3081,))
%             ])
%             if target == 'combined':
%                 x_train = np.load(f'data/distribution_colored_EMNIST_Seed42_Var{variance}_Corr{correlation}/x_train_colored.npy')
%                 y_train_combined = np.load(f'data/distribution_colored_EMNIST_Seed42_Var{variance}_Corr{correlation}/y_train_combined.npy')
%                 x_test = np.load('data/colored_EMNIST/x_test_colored.npy')
%                 y_test_combined = np.load('data/colored_EMNIST/y_test_combined.npy')
                
%                 # Apply transformation
%                 x_train_tensor = apply_transform(x_train, transform)
%                 x_test_tensor = apply_transform(x_test, transform)
                
%                 y_train_tensor = torch.tensor(y_train_combined, dtype=torch.long)
%                 y_test_tensor = torch.tensor(y_test_combined, dtype=torch.long)
                
%                 train_dataset = torch.utils.data.TensorDataset(x_train_tensor, y_train_tensor)
%                 test_dataset = torch.utils.data.TensorDataset(x_test_tensor, y_test_tensor)
        
%             num_classes = 10 if target in ['color', 'digit'] else 100
%             in_channels = 3
%             imagesize = (32, 32)
%         else:
%             raise ValueError("Invalid dataset name")
            
    
%         if gray_scale:
%             transform = transforms.Compose([
%                 transforms.Grayscale(),
%                 transforms.ToTensor(),
%                 transforms.Normalize((0.1307,), (0.3081,))
%             ])
%             train_dataset.transform = transform
%             test_dataset.transform = transform  
    
%         return train_dataset, test_dataset, imagesize, num_classes, in_channels
    
%     def load_models(in_channels, args, img_size, num_classes):
%         if args.model == "cnn_2layers":
%             model = CNN2Layer(in_channels, num_classes, args.model_width, img_size)
%         elif args.model == "cnn_3layers":
%             model = CNN3Layer(in_channels, num_classes, args.model_width, img_size)
%         elif args.model == "cnn_4layers":
%             model = CNN4Layer(in_channels, num_classes, args.model_width, img_size)
%         elif args.model == "cnn_5layers":
%             model = CNN5Layer(in_channels, num_classes, args.model_width, img_size)
%         elif args.model == "cnn_8layers":
%             model = CNN8Layer(in_channels, num_classes, args.model_width, img_size)
%         elif args.model == "cnn_16layers":
%             model = CNN16Layer(in_channels, num_classes, args.model_width, img_size)
%         elif args.model == "resnet18":
%             model = models.resnet18(num_classes=num_classes)
%         else:
%             raise ValueError("Invalid model name.")
%         return model
    
%     # Modify add_label_noise to return indices of noisy samples
%     def add_label_noise(targets, label_noise_rate, num_digits, num_colors):
%         noisy_targets = targets.clone()
%         num_noisy = int(label_noise_rate * len(targets))
%         noisy_indices = torch.randperm(len(targets))[:num_noisy]
%         noise_info = torch.zeros(len(targets), dtype=torch.int)  # Initialize as clean
    
%         if num_digits == 10 and num_colors == 1:
%             for idx in noisy_indices:
%                 original_label = targets[idx].item()
%                 new_label = random.randint(0, num_digits - 1)
%                 while new_label == original_label:
%                     new_label = random.randint(0, num_digits - 1)
%                 noisy_targets[idx] = new_label
%                 noise_info[idx] = 1  # Mark as noisy
    
%         elif num_digits == 10 and num_colors == 10:
%             for idx in noisy_indices:
%                 original_label = targets[idx].item()
%                 original_digit = original_label // num_colors
%                 original_color = original_label % num_colors
    
%                 new_digit = random.randint(0, num_digits - 1)
%                 new_color = random.randint(0, num_colors - 1)
%                 new_label = new_digit * num_colors + new_color
%                 while new_label == original_label:
%                     new_digit = random.randint(0, num_digits - 1)
%                     new_color = random.randint(0, num_colors - 1)
%                     new_label = new_digit * num_colors + new_color
    
%                 noisy_targets[idx] = new_label
%                 noise_info[idx] = 1  # Mark as noisy
    
%         return noisy_targets, noise_info
    
%     def train_model(model, train_loader, optimizer, criterion, weight_noisy, weight_clean, device, num_colors, num_digits):
%         """
%         Training function with comprehensive metrics tracking.
        
%         Args:
%             model: The neural network model
%             train_loader: DataLoader for training data
%             optimizer: The optimizer
%             criterion: Loss function (expected to support reduction='none')
%             weight_noisy: Weight for noisy samples
%             weight_clean: Weight for clean samples
%             device: Device to run the training on
%             num_colors: Number of color classes
%             num_digits: Number of digit classes
            
%         Returns:
%             dict: Dictionary containing all training metrics
%         """
%         model.train()
%         running_loss = 0.0
%         total_samples = 0
%         correct_total = 0
    
%         # Initialize counters for noisy and clean samples
%         correct_noisy = 0
%         total_noisy = 0
%         correct_clean = 0
%         total_clean = 0
    
%         # Initialize counters for digits and colors
%         correct_digit_total = 0
%         correct_color_total = 0
    
%         correct_digit_noisy = 0
%         correct_color_noisy = 0
%         correct_digit_clean = 0
%         correct_color_clean = 0
    
%         # Lists for loss tracking
%         loss_values = []
%         loss_values_noisy = []
%         loss_values_clean = []
    
%         # Ensure criterion returns per-sample losses
%         criterion.reduction = 'none'
    
%         for inputs, labels, noise_labels in train_loader:
%             try:
%                 # Move data to device
%                 inputs = inputs.to(device, non_blocking=True)
%                 labels = labels.to(device, non_blocking=True)
%                 noise_labels = noise_labels.to(device, non_blocking=True)
    
%                 # Zero the parameter gradients
%                 optimizer.zero_grad()
    
%                 # Forward pass
%                 outputs = model(inputs)
%                 _, predicted = torch.max(outputs.data, 1)
                
%                 # Calculate batch size and update total samples
%                 batch_size = labels.size(0)
%                 total_samples += batch_size
    
%                 # Get indices for noisy and clean samples
%                 idx_noisy = (noise_labels == 1)
%                 idx_clean = (noise_labels == 0)
    
%                 # Count noisy and clean samples in batch
%                 num_noisy = idx_noisy.sum().item()
%                 num_clean = idx_clean.sum().item()
    
%                 # Compute per-sample losses
%                 per_sample_loss = criterion(outputs, labels)
    
%                 # Calculate weights for the batch
%                 total_weight = weight_clean + weight_noisy
%                 weights = torch.zeros_like(per_sample_loss, device=device)
                
%                 if num_noisy == 0:  # All clean samples
%                     weights = torch.ones_like(per_sample_loss, device=device) * (weight_clean / total_weight) * 2
%                 elif num_clean == 0:  # All noisy samples
%                     weights = torch.ones_like(per_sample_loss, device=device) * (weight_noisy / total_weight) * 2
%                 else:  # Mixed batch
%                     weights[idx_noisy] = (weight_noisy / total_weight) * 2
%                     weights[idx_clean] = (weight_clean / total_weight) * 2
    
%                 # Apply weights to losses
%                 per_sample_loss_weighted = per_sample_loss * weights
    
%                 # Compute mean loss and backpropagate
%                 loss = per_sample_loss_weighted.mean()
%                 loss.backward()
%                 optimizer.step()
    
%                 # Update running loss and total accuracy
%                 running_loss += loss.item() * batch_size
%                 correct_total += (predicted == labels).sum().item()
    
%                 digit_labels = labels // num_colors
%                 color_labels = labels % num_colors
%                 digit_predictions = predicted // num_colors
%                 color_predictions = predicted % num_colors
    
%                 correct_digit_total += (digit_predictions == digit_labels).sum().item()
%                 correct_color_total += (color_predictions == color_labels).sum().item()
    
%                 # Process noisy samples
%                 if num_noisy > 0:
%                     labels_noisy = labels[idx_noisy]
%                     predicted_noisy = predicted[idx_noisy]
%                     correct_noisy += (predicted_noisy == labels_noisy).sum().item()
%                     total_noisy += num_noisy
    
%                     # Update digit and color accuracy for noisy samples
%                     digit_labels_noisy = labels_noisy // num_colors
%                     color_labels_noisy = labels_noisy % num_colors
%                     digit_predictions_noisy = predicted_noisy // num_colors
%                     color_predictions_noisy = predicted_noisy % num_colors
    
%                     correct_digit_noisy += (digit_predictions_noisy == digit_labels_noisy).sum().item()
%                     correct_color_noisy += (color_predictions_noisy == color_labels_noisy).sum().item()
    
%                     # Store noisy sample losses
%                     loss_values_noisy.extend(per_sample_loss_weighted[idx_noisy].detach().cpu().numpy())
    
%                 # Process clean samples
%                 if num_clean > 0:
%                     labels_clean = labels[idx_clean]
%                     predicted_clean = predicted[idx_clean]
%                     correct_clean += (predicted_clean == labels_clean).sum().item()
%                     total_clean += num_clean
    
%                     # Update digit and color accuracy for clean samples
%                     digit_labels_clean = labels_clean // num_colors
%                     color_labels_clean = labels_clean % num_colors
%                     digit_predictions_clean = predicted_clean // num_colors
%                     color_predictions_clean = predicted_clean % num_colors
    
%                     correct_digit_clean += (digit_predictions_clean == digit_labels_clean).sum().item()
%                     correct_color_clean += (color_predictions_clean == color_labels_clean).sum().item()
    
%                     # Store clean sample losses
%                     loss_values_clean.extend(per_sample_loss_weighted[idx_clean].detach().cpu().numpy())
    
%                 # Store all losses
%                 loss_values.extend(per_sample_loss_weighted.detach().cpu().numpy())
    
%             except Exception as e:
%                 print(f"Error in training batch: {str(e)}")
%                 continue
    
%         # Calculate loss statistics
%         avg_loss = np.mean(loss_values) if loss_values else float('nan')
%         var_loss = np.var(loss_values) if loss_values else float('nan')
    
%         avg_loss_noisy = np.mean(loss_values_noisy) if loss_values_noisy else float('nan')
%         var_loss_noisy = np.var(loss_values_noisy) if loss_values_noisy else float('nan')
    
%         avg_loss_clean = np.mean(loss_values_clean) if loss_values_clean else float('nan')
%         var_loss_clean = np.var(loss_values_clean) if loss_values_clean else float('nan')
    
%         # Calculate accuracies
%         metrics = {
%             'avg_loss': avg_loss,
%             'var_loss': var_loss,
%             'accuracy_total': 100. * correct_total / total_samples if total_samples > 0 else float('nan'),
%             'accuracy_noisy': 100. * correct_noisy / total_noisy if total_noisy > 0 else float('nan'),
%             'accuracy_clean': 100. * correct_clean / total_clean if total_clean > 0 else float('nan'),
%             'avg_loss_noisy': avg_loss_noisy,
%             'var_loss_noisy': var_loss_noisy,
%             'avg_loss_clean': avg_loss_clean,
%             'var_loss_clean': var_loss_clean,
%             'accuracy_digit_total': 100. * correct_digit_total / total_samples if total_samples > 0 else float('nan'),
%             'accuracy_color_total': 100. * correct_color_total / total_samples if total_samples > 0 else float('nan'),
%             'accuracy_digit_noisy': 100. * correct_digit_noisy / total_noisy if total_noisy > 0 else float('nan'),
%             'accuracy_color_noisy': 100. * correct_color_noisy / total_noisy if total_noisy > 0 else float('nan'),
%             'accuracy_digit_clean': 100. * correct_digit_clean / total_clean if total_clean > 0 else float('nan'),
%             'accuracy_color_clean': 100. * correct_color_clean / total_clean if total_clean > 0 else float('nan'),
%             'total_samples': total_samples,
%             'total_noisy': total_noisy,
%             'total_clean': total_clean,
%             'correct_total': correct_total,
%             'correct_noisy': correct_noisy,
%             'correct_clean': correct_clean
%         }
    
%         return metrics
    
%     def test_model(model, test_loader, device, num_colors, num_digits):
%         model.eval()
%         loss_values = []  # 損失値を保存するリスト
%         test_loss = 0
%         correct_total = 0
%         total_samples = 0
    
%         correct_digit_total = 0
%         correct_color_total = 0
    
%         # criterion を 'none' に設定して個々のサンプルの損失を取得
%         criterion = nn.CrossEntropyLoss(reduction='none')
    
%         with torch.no_grad():
%             for inputs, labels in test_loader:
%                 inputs, labels = inputs.to(device), labels.to(device)
%                 outputs = model(inputs)
                
%                 # 個々のサンプルの損失を計算
%                 per_sample_loss = criterion(outputs, labels)
%                 loss_values.extend(per_sample_loss.cpu().numpy())
                
%                 # バッチ全体の損失を計算
%                 test_loss += per_sample_loss.mean().item() * labels.size(0)
                
%                 _, predicted = torch.max(outputs, 1)
%                 total_samples += labels.size(0)
%                 correct_total += (predicted == labels).sum().item()
    
%                 # Calculate correct counts for digits and colors
%                 digit_labels = labels // num_colors
%                 color_labels = labels % num_colors
%                 digit_predictions = predicted // num_colors
%                 color_predictions = predicted % num_colors
    
%                 correct_digit_total += (digit_predictions == digit_labels).sum().item()
%                 correct_color_total += (color_predictions == color_labels).sum().item()
    
%         # 損失の平均と分散を計算
%         avg_loss = test_loss / total_samples if total_samples > 0 else float('nan')
%         var_loss = np.var(loss_values) if loss_values else float('nan')
        
%         accuracy_total = 100. * correct_total / total_samples if total_samples > 0 else float('nan')
%         accuracy_digit_total = 100. * correct_digit_total / total_samples if total_samples > 0 else float('nan')
%         accuracy_color_total = 100. * correct_color_total / total_samples if total_samples > 0 else float('nan')
    
%         return {
%             'avg_loss': avg_loss,
%             'var_loss': var_loss,  # 追加
%             'accuracy_total': accuracy_total,
%             'accuracy_digit_total': accuracy_digit_total,
%             'accuracy_color_total': accuracy_color_total,
%             'total_samples': total_samples,
%             'correct_total': correct_total
%         }
        
%     def main():
%         """
%         Main training loop with comprehensive error handling and logging
%         """
%         print('Start session')
%         wandb_run = None
        
%         try:
%             # Parse arguments and set initial configurations
%             args = parse_args()
%             set_seed(args.fix_seed)
            
%             # Set device
%             device = set_device(args.gpu)
%             print(f'Using device: {device}')
    
%             # Load datasets with error handling
%             print('Loading datasets...')
%             try:
%                 train_dataset, test_dataset, imagesize, num_classes, in_channels = load_datasets(
%                     args.dataset, args.target, args.gray_scale, args)
%             except FileNotFoundError as e:
%                 print(f"Error loading dataset: {e}")
%                 return
%             except Exception as e:
%                 print(f"Unexpected error loading dataset: {e}")
%                 return
    
%             # Set number of digit and color classes
%             if args.target == 'combined':
%                 num_digits = 10
%                 num_colors = 10
%             else:
%                 num_digits = 10
%                 num_colors = 1
    
%             # Add label noise and create NoisyDataset
%             print(f'Preparing dataset with label noise rate: {args.label_noise_rate}')
%             if args.label_noise_rate > 0.0:
%                 if hasattr(train_dataset, 'tensors'):
%                     x_train, y_train = train_dataset.tensors
%                     y_train_noisy, noise_info = add_label_noise(y_train, args.label_noise_rate, num_digits, num_colors)
%                     train_dataset = torch.utils.data.TensorDataset(x_train, y_train_noisy)
%                     train_dataset = NoisyDataset(train_dataset, noise_info)
%                 else:
%                     y_train = torch.tensor(train_dataset.targets)
%                     y_train_noisy, noise_info = add_label_noise(y_train, args.label_noise_rate, num_digits, num_colors)
%                     train_dataset.targets = y_train_noisy.tolist()
%                     train_dataset = NoisyDataset(train_dataset, noise_info)
%             else:
%                 noise_info = torch.zeros(len(train_dataset), dtype=torch.int)
%                 train_dataset = NoisyDataset(train_dataset, noise_info)
    
%             # Extract indices for clean and noisy samples
%             clean_indices = [i for i, label in enumerate(noise_info) if label == 0]
%             noisy_indices = [i for i, label in enumerate(noise_info) if label == 1]
    
%             # Validate batch size
%             if args.batch_size % 2 != 0:
%                 raise ValueError("Batch size must be even for balanced batches")
    
%             # Create data loaders
%             print('Setting up data loaders...')
%             if args.label_noise_rate == 0.0 or args.label_noise_rate == 1.0:
%                 train_loader = torch.utils.data.DataLoader(
%                     train_dataset,
%                     batch_size=args.batch_size,
%                     shuffle=True,
%                     num_workers=args.num_workers,
%                     pin_memory=True
%                 )
%             else:
%                 batch_sampler = BalancedBatchSampler(
%                     clean_indices,
%                     noisy_indices,
%                     args.batch_size,
%                     drop_last=False
%                 )
%                 train_loader = torch.utils.data.DataLoader(
%                     train_dataset,
%                     batch_sampler=batch_sampler,
%                     num_workers=args.num_workers,
%                     pin_memory=True
%                 )
    
%             test_loader = torch.utils.data.DataLoader(
%                 test_dataset,
%                 batch_size=args.batch_size,
%                 shuffle=False,
%                 num_workers=args.num_workers,
%                 pin_memory=True
%             )
    
%             # Initialize model
%             print('Initializing model...')
%             model = load_models(in_channels, args, imagesize, num_classes)
%             model = model.to(device)
    
%             # Set optimizer
%             print('Setting up optimizer...')
%             if args.optimizer == "sgd":
%                 optimizer = optim.SGD(model.parameters(), lr=args.lr, momentum=args.momentum)
%             elif args.optimizer == "adam":
%                 optimizer = optim.Adam(model.parameters(), lr=args.lr)
%             elif args.optimizer == "adamw":
%                 optimizer = optim.AdamW(model.parameters(), lr=args.lr)
%             elif args.optimizer == "rmsprop":
%                 optimizer = optim.RMSprop(model.parameters(), lr=args.lr, momentum=args.momentum)
%             elif args.optimizer == "adagrad":
%                 optimizer = optim.Adagrad(model.parameters(), lr=args.lr)
%             else:
%                 raise ValueError(f"Unsupported optimizer: {args.optimizer}")
    
%             # Set loss function
%             criterion = nn.CrossEntropyLoss(reduction='none')
    
%             # Set experiment name
%             experiment_name = (
%                 f'{args.model}_{args.dataset}_{args.target}_'
%                 f'lr{args.lr}_batch{args.batch_size}_epoch{args.epoch}_'
%                 f'LabelNoiseRate{args.label_noise_rate}_Optim{args.optimizer}_'
%                 f'cleanw{args.weight_clean}_noisew{args.weight_noisy}_variance{args.variance}_width{args.model_width}_'
%                 f'seed{args.fix_seed}'
                
%             )
%             print(f'Experiment name: {experiment_name}')
    
%             # Initialize wandb
%             if args.wandb:
%                 print('Initializing wandb...')
%                 wandb_run = wandb.init(
%                     project=args.wandb_project,
%                     name=experiment_name,
%                     entity=args.wandb_entity,
%                     config=args
%                 )
    
%             # Set up CSV logging
%             csv_dir = f"./csv/combine/split_noise/{experiment_name}"
%             os.makedirs(csv_dir, exist_ok=True)
%             csv_path = os.path.join(csv_dir, 'log.csv')
            
%             if not os.path.isfile(csv_path):
%                 with open(csv_path, 'w', newline='') as f:
%                     writer = csv.writer(f)
%                     writer.writerow([
%                         "epoch", 
%                         "train_loss", "train_loss_variance",
%                         "train_accuracy", "train_accuracy_noisy", "train_accuracy_clean",
%                         "train_accuracy_digit_total", "train_accuracy_color_total",
%                         "train_accuracy_digit_noisy", "train_accuracy_color_noisy",
%                         "train_accuracy_digit_clean", "train_accuracy_color_clean",
%                         "test_loss", "test_loss_variance",
%                         "test_accuracy",
%                         "test_accuracy_digit_total", "test_accuracy_color_total"
%                     ])
    
%             # Training loop
%             print('Starting training...')
%             best_test_accuracy = 0.0
%             for epoch in range(1, args.epoch + 1):
%                 try:
%                     # Training phase
%                     train_metrics = train_model(
%                         model=model,
%                         train_loader=train_loader,
%                         optimizer=optimizer,
%                         criterion=criterion,
%                         weight_noisy=args.weight_noisy,
%                         weight_clean=args.weight_clean,
%                         device=device,
%                         num_colors=num_colors,
%                         num_digits=num_digits
%                     )
    
%                     # Testing phase
%                     test_metrics = test_model(
%                         model=model,
%                         test_loader=test_loader,
%                         device=device,
%                         num_colors=num_colors,
%                         num_digits=num_digits
%                     )
    
%                     # Print progress
%                     print(f"\nEpoch: {epoch}/{args.epoch}")
%                     print(f"Train Loss: {train_metrics['avg_loss']:.4f}, "
%                           f"Train Loss Variance: {train_metrics['var_loss']:.4f}")
%                     print(f"Train Accuracy: {train_metrics['accuracy_total']:.2f}%, "
%                           f"Test Accuracy: {test_metrics['accuracy_total']:.2f}%")
%                     print(f"Train Digit/Color Accuracy: {train_metrics['accuracy_digit_total']:.2f}%/"
%                           f"{train_metrics['accuracy_color_total']:.2f}%")
%                     print(f"Test Digit/Color Accuracy: {test_metrics['accuracy_digit_total']:.2f}%/"
%                           f"{test_metrics['accuracy_color_total']:.2f}%")
    
%                     # Save to CSV
%                     with open(csv_path, 'a', newline='') as f:
%                         writer = csv.writer(f)
%                         writer.writerow([
%                             epoch,
%                             train_metrics['avg_loss'], train_metrics['var_loss'],
%                             train_metrics['accuracy_total'],
%                             train_metrics['accuracy_noisy'],
%                             train_metrics['accuracy_clean'],
%                             train_metrics['accuracy_digit_total'],
%                             train_metrics['accuracy_color_total'],
%                             train_metrics['accuracy_digit_noisy'],
%                             train_metrics['accuracy_color_noisy'],
%                             train_metrics['accuracy_digit_clean'],
%                             train_metrics['accuracy_color_clean'],
%                             test_metrics['avg_loss'], test_metrics['var_loss'],
%                             test_metrics['accuracy_total'],
%                             test_metrics['accuracy_digit_total'],
%                             test_metrics['accuracy_color_total']
%                         ])
    
%                     # Log to wandb
%                     if args.wandb:
%                         wandb.log({
%                             'epoch': epoch,
%                             'train_loss': train_metrics['avg_loss'],
%                             'train_loss_variance': train_metrics['var_loss'],
%                             'train_accuracy': train_metrics['accuracy_total'],
%                             'train_accuracy_noisy': train_metrics['accuracy_noisy'],
%                             'train_accuracy_clean': train_metrics['accuracy_clean'],
%                             'train_accuracy_digit_total': train_metrics['accuracy_digit_total'],
%                             'train_accuracy_color_total': train_metrics['accuracy_color_total'],
%                             'train_accuracy_digit_noisy': train_metrics['accuracy_digit_noisy'],
%                             'train_accuracy_color_noisy': train_metrics['accuracy_color_noisy'],
%                             'train_accuracy_digit_clean': train_metrics['accuracy_digit_clean'],
%                             'train_accuracy_color_clean': train_metrics['accuracy_color_clean'],
%                             'test_loss': test_metrics['avg_loss'],
%                             'test_loss_variance': test_metrics['var_loss'],
%                             'test_accuracy': test_metrics['accuracy_total'],
%                             'test_accuracy_digit_total': test_metrics['accuracy_digit_total'],
%                             'test_accuracy_color_total': test_metrics['accuracy_color_total']
%                         })
    
%                     # Memory management
%                     if epoch % 10 == 0:
%                         clear_memory()
    
%                     # Save best model
%                     if test_metrics['accuracy_total'] > best_test_accuracy:
%                         best_test_accuracy = test_metrics['accuracy_total']
%                         torch.save(model.state_dict(), 
%                                  os.path.join(csv_dir, 'best_model.pth'))
    
%                 except Exception as e:
%                     print(f"Error in epoch {epoch}: {str(e)}")
%                     continue
    
%         except Exception as e:
%             print(f"Fatal error in training: {str(e)}")
%             raise
        
%         finally:
%             # Cleanup
%             if wandb_run is not None:
%                 wandb_run.finish()
%             print('Training completed')
    
%     if __name__ == '__main__':
%         main()
% \end{lstlisting}
