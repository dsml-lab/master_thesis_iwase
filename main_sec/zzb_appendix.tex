% \chapter*{付録}
% \addcontentsline{toc}{chapter}{付録}
\chapter{学習プログラム}
\label{学習プログラム}
学習に利用したプログラムを示す.掲載したプログラムは,複数あるバージョンの一つである.
{\tiny
\begin{minted}[linenos=true,frame=single,numbersep=6pt,mathescape=true,breaklines]{python}
import torch
import torch.nn as nn
import torch.optim as optim
import torch.nn.functional as f
import torch.backends.cudnn as cudnn
import torchvision
import torchvision.transforms as transforms;
from sklearn.metrics import classification_report
from sklearn.metrics import accuracy_score
import math
from model import resnet18k 
import torchvision.models as models
# from transformers import ViTForImageClassification

import argparse
import random
import matplotlib.pyplot as plt
import csv 
import warnings
import os
import sys
import wandb
import numpy as np
import cv2
from PIL import Image

def get_model(args):
    """モデルの読み込み

    Arguments:
        args: 
    Returns:
        model
    """
    if args.model == "resnet18k":
        assert args.model_width is not None, "please check k value"
        model = resnet18k.make_resnet18k(k=args.model_width, num_classes=args.num_classes)
        args.model_fullname = "SR_resnet18k-{}".format(args.model_width)
    elif args.model == "resnet18":
        if args.pretrained == "SR":
            # model
            model = models.resnet18(pretrained=False, num_classes=args.num_classes)
            args.model_fullname = "SR_resnet18" # SR means scratch
        elif args.pretrained == "IN": ## 事前学習モデルを利用する場合,一度重みを読み込んで,全結合層だけ初期化する
            model = models.resnet18(pretrained=True)
            in_features = model.fc.in_features
            model.fc = nn.Linear(in_features=in_features, out_features=args.num_classes, bias=True)
            args.model_fullname = "IN_resnet18"
        elif args.pretrained == "INv1": ## 独自に学習したバージョン、バッチサイズ256
            model = models.resnet18()
            model.load_state_dict(
            torch.load('/workspace/Epoch_dd/checkpoint/SR_resnet18_90epochs_bs256_ImageNet_ln0pc.tar')["model_state_dict"]
            )
            in_features = model.fc.in_features
            model.fc = nn.Linear(in_features=in_features, out_features=args.num_classes, bias=True)
            args.model_fullname = "INv1_resnet18"
        elif args.pretrained == "INv2": ## 独自に学習したバージョン、バッチサイズ256
            model = models.resnet18()
            model.load_state_dict(
            torch.load('/workspace/Epoch_dd/checkpoint/SR_resnet18_90epochs_bs32_ImageNet_ln0pc.tar')["model_state_dict"]
            )
            in_features = model.fc.in_features
            model.fc = nn.Linear(in_features=in_features, out_features=args.num_classes, bias=True)
            args.model_fullname = "INv2_resnet18"
        elif args.pretrained == "INv3": ## pytorchのプログラムで事前学習
            model = models.resnet18()
            model.load_state_dict(
            torch.load('/workspace/Epoch_dd/checkpoint/INv3_made_by_pytorch_code.pth')["model"]
            )
            in_features = model.fc.in_features
            model.fc = nn.Linear(in_features=in_features, out_features=args.num_classes, bias=True)
            args.model_fullname = "INv3_resnet18"
        elif args.pretrained == "INv4": ## pytorchのプログラムで事前学習
            model = models.resnet18()
            model.load_state_dict(
            torch.load('/workspace/Epoch_dd/checkpoint/SR_resnet18_90epochs_bs32_ImageNet_ln0pc_v2.tar')["model_state_dict"]
            )
            in_features = model.fc.in_features
            model.fc = nn.Linear(in_features=in_features, out_features=args.num_classes, bias=True)
            args.model_fullname = "INv4_resnet18"
        elif args.pretrained == "FDB1kv1": ## 独自に学習したバージョン、バッチサイズ256
            model = models.resnet18()
            model.load_state_dict(
            torch.load('/workspace/Epoch_dd/checkpoint/SR_resnet18_90epochs_bs32_FDB_1k_ln0pc.tar')["model_state_dict"]
            )
            in_features = model.fc.in_features
            model.fc = nn.Linear(in_features=in_features, out_features=args.num_classes, bias=True)
            args.model_fullname = "FDB1kv1_resnet18"
        elif args.pretrained == "RCDB1kv1": ## 独自に学習したバージョン、バッチサイズ256
            model = models.resnet18()
            model.load_state_dict(
            torch.load('/workspace/Epoch_dd/checkpoint/SR_resnet18_90epochs_bs32_RCDB1k_ln0pc.tar')["model_state_dict"]
            )
            in_features = model.fc.in_features
            model.fc = nn.Linear(in_features=in_features, out_features=args.num_classes, bias=True)
            args.model_fullname = "RCDB1kv1_resnet18"
        elif args.pretrained == "FR":
            model = models.resnet18()
            model.load_state_dict(torch.load('/workspace/Epoch_dd/model_weight/Fractal-1k/FractalDB-1000_res18.pth'))
            in_features = model.fc.in_features
            model.fc = nn.Linear(in_features=in_features, out_features=args.num_classes, bias=True)
            args.model_fullname = "FR_resnet18"
    elif args.model == "resnet34":
        if args.pretrained == "SR":
            # model
            model = models.resnet34(pretrained=False, num_classes=args.num_classes)
            args.model_fullname = "SR_resnet34" # SR means scratch
        else: ## 事前学習モデルを利用する場合,一度重みを読み込んで,全結合層だけ初期化する
            model = models.resnet34(pretrained=True)
            in_features = model.fc.in_features
            model.fc = nn.Linear(in_features=in_features, out_features=args.num_classes, bias=True)
            args.model_fullname = "IN_resnet34"
    elif args.model == "resnet50":
        if args.pretrained == "SR":
            # model
            model = models.resnet50(pretrained=False, num_classes=args.num_classes)
            args.model_fullname = "SR_resnet50" # SR means scratch
        else: ## 事前学習モデルを利用する場合,一度重みを読み込んで,全結合層だけ初期化する
            model = models.resnet50(pretrained=True)
            in_features = model.fc.in_features
            model.fc = nn.Linear(in_features=in_features, out_features=args.num_classes, bias=True)
            args.model_fullname = "IN_resnet50"
    elif args.model == "densenet121":
        if args.pretrained == "SR":
            # model
            model = models.densenet121(pretrained=False, num_classes=args.num_classes)
            args.model_fullname = "SR_densenet121" # SR means scratch
        else: ## 事前学習モデルを利用する場合,一度重みを読み込んで,全結合層だけ初期化する
            model = models.densenet121(pretrained=True)
            in_features = model.classifier.in_features
            model.classifier = nn.Linear(in_features=in_features, out_features=args.num_classes, bias=True)
            args.model_fullname = "IN_densenet121"
    elif args.model == "mobilenetV2":
        if args.pretrained == "SR":
            # model
            model = models.mobilenet_v2(pretrained=False, num_classes=args.num_classes)
            args.model_fullname = "SR_mobilenetV2" # SR means scratch
        else: ## 事前学習モデルを利用する場合,一度重みを読み込んで,全結合層だけ初期化する
            model = models.mobilenet_v2(pretrained=True)
            in_features = model.classifier[1].in_features
            model.classifier[1] = nn.Linear(in_features=in_features, out_features=args.num_classes, bias=True)
            args.model_fullname = "IN_mobilenetV2"
    elif args.model == "vitB16":
        if args.pretrained == "SR":
            # model
            model = models.vit_b_16(pretrained=False, num_classes=args.num_classes)
            args.model_fullname = "SR_vitB16" # SR means scratch
        else: 
            model = models.vit_b_16(pretrained=True)
            in_features = model.heads[0].in_features
            model.heads[0] = nn.Linear(in_features=in_features, out_features=args.num_classes)
            args.model_fullname = "IN_vitB16"
    return model

def get_dataset_train(args, transform):
    if args.dataset == "cifar10":
        return torchvision.datasets.CIFAR10(root='./data', train=True, download=True, transform=transform)
    elif args.dataset == "cannyShapeCifar10ColorV1":
        train_set = torchvision.datasets.CIFAR10(root='./data', train=True, download=True, transform=transform)
        # RGBで形状を取り出して代入
        for i in range(len(train_set)):
            new_image = np.array(train_set.data[i], dtype=np.uint8)
            new_image = cv2.cvtColor(new_image, cv2.COLOR_RGB2BGR)
            edges = cv2.Canny(image=new_image,threshold1=255 / 2, threshold2=255)
            edges = cv2.bitwise_and(new_image, new_image, mask=edges)
            train_set.data[i] = Image.fromarray(edges)
        return train_set
    elif args.dataset == "cifar100":
        return torchvision.datasets.CIFAR100(root='./data', train=True, download=True, transform=transform)
    elif args.dataset == "tinyImageNet":
        return torchvision.datasets.ImageFolder(root='./data/tiny-imagenet-200/train', transform=transform)
    elif args.dataset == "DTD":
        return torchvision.datasets.DTD(root='./data', split="train", download=True, transform=transform)
    elif args.dataset == "STL10":
        return torchvision.datasets.STL10(root='./data', split="train", download=True, transform=transform)
    elif args.dataset == "flower102":
        return torchvision.datasets.Flowers102(root='./data', split="train", download=True, transform=transform)
    elif args.dataset == "cifar100_to_10class":
        selected_classes = list(range(10))
        train_set = torchvision.datasets.CIFAR100(root='./data', train=True, download=True, transform=transform)
        targets = torch.tensor(train_set.targets)
    
        # 選択したクラスに対応するブーリアンマスクを作成
        mask = torch.zeros_like(targets, dtype=torch.bool)
        for c in selected_classes:
            mask |= (targets == c)
        
        # 選択したクラスに対応するデータとターゲットを抽出
        train_set.data = train_set.data[mask]
        train_set.targets = targets[mask].tolist()
        return train_set
        
def get_dataset_test(args, transform):
    if args.dataset == "cifar10":
        return torchvision.datasets.CIFAR10(root='./data', train=False, download=True, transform=transform)
    elif args.dataset == "cannyShapeCifar10ColorV1":
        test_set = torchvision.datasets.CIFAR10(root='./data', train=False, download=True, transform=transform)
        # # RGBで形状を取り出して代入
        # for i in range(len(test_set)):
        #     new_image = np.array(test_set.data[i], dtype=np.uint8)
        #     new_image = cv2.cvtColor(new_image, cv2.COLOR_RGB2BGR)
        #     edges = cv2.Canny(image=new_image,threshold1=255 / 2, threshold2=255)
        #     edges = cv2.bitwise_and(new_image, new_image, mask=edges)
        #     test_set.data[i] = Image.fromarray(edges)
        return test_set
    elif args.dataset == "cifar100":
        return torchvision.datasets.CIFAR100(root='./data', train=False, download=True, transform=transform)
    elif args.dataset == "tinyImageNet":
        return torchvision.datasets.ImageFolder(root='./data/tiny-imagenet-200/val', transform=transform)
    elif args.dataset == "DTD":
        return torchvision.datasets.DTD(root='./data', split="test", download=True, transform=transform)
    elif args.dataset == "STL10":
        return torchvision.datasets.STL10(root='./data', split="test", download=True, transform=transform)
    elif args.dataset == "flower102":
        return torchvision.datasets.Flowers102(root='./data', split="test", download=True, transform=transform)
    elif args.dataset == "cifar100_to_10class":
        selected_classes = list(range(10))
        test_set = torchvision.datasets.CIFAR100(root='./data', train=False, download=True, transform=transform)
        targets = torch.tensor(test_set.targets)
    
        # 選択したクラスに対応するブーリアンマスクを作成
        mask = torch.zeros_like(targets, dtype=torch.bool)
        for c in selected_classes:
            mask |= (targets == c)
        
        # 選択したクラスに対応するデータとターゲットを抽出
        test_set.data = test_set.data[mask]
        test_set.targets = targets[mask].tolist()
        return test_set

def get_imagesize(args):
    if args.dataset == "cifar10":
        return 32
    elif args.dataset == "cannyShapeCifar10ColorV1":
        return 32
    elif args.dataset == "cifar100":
        return 32
    elif args.dataset == "tinyImageNet":
        return 64
    elif args.dataset == "DTD":
        return 224
    elif args.dataset == "STL10":
        return 96
    elif args.dataset == "flower102":
        return 224
    elif args.dataset == "cifar100_to_10class":
        return 32

def get_num_classes(args):
    if args.dataset == "cifar10":
        return 10
    elif args.dataset == "cannyShapeCifar10ColorV1":
        return 10
    elif args.dataset == "cifar100":
        return 100
    elif args.dataset == "tinyImageNet":
        return 200
    elif args.dataset == "DTD":
        return 47
    elif args.dataset == "STL10":
        return 10
    elif args.dataset == "flower102":
        return 102
    elif args.dataset == "cifar100_to_10class":
        return 10
    

def fix_seed(seed=42):
    # random
    random.seed(seed)
    # Numpy
    # np.random.seed(seed)
    # Pytorch
    torch.manual_seed(seed) 
    torch.cuda.manual_seed_all(seed)
    torch.backends.cudnn.benchmark = False
    torch.backends.cudnn.deterministic = True
    # Tensorflow
    # tf.random.set_seed(seed)

def main():
    args = parse_args()
    fix_seed(args.fix_seed)
    #epoch数指定
    epoch = args.epoch
    label_noise_rate = args.label_noise_rate
    args.num_classes = get_num_classes(args)
    device = 'cuda' if torch.cuda.is_available() else 'cpu'
    imageSize = get_imagesize(args)

    if args.model == "vitB16" or args.dataset == "DTD":
        args.resize = True

    if args.resize:
        transform_train = transforms.Compose([
            transforms.RandomCrop(imageSize, padding=imageSize // 8),
            transforms.RandomHorizontalFlip(),
            transforms.ToTensor(),
            transforms.Normalize((0.4914, 0.4822, 0.4465), (0.2023, 0.1994, 0.2010)),
            transforms.Resize(size=(224, 224)),
        ])
        transform_test = transforms.Compose([
            transforms.ToTensor(),
            transforms.Normalize((0.4914, 0.4822, 0.4465), (0.2023, 0.1994, 0.2010)),
            transforms.Resize(size=(224, 224)),
        ])
    else:
        transform_train = transforms.Compose([
            transforms.RandomCrop(imageSize, padding=imageSize // 8),
            transforms.RandomHorizontalFlip(),
            transforms.ToTensor(),
            transforms.Normalize((0.4914, 0.4822, 0.4465), (0.2023, 0.1994, 0.2010)),
        ])
        transform_test = transforms.Compose([
            transforms.ToTensor(),
            transforms.Normalize((0.4914, 0.4822, 0.4465), (0.2023, 0.1994, 0.2010)),
        ])
    #resize分岐
    train_set = get_dataset_train(args, transform_train)

    #インスタンス変数にアクセスしてラベルの張替え
    if hasattr(train_set, "targets"):
        for i in range(len(train_set.targets)):
            if(random.randint(0, 9999) < int(label_noise_rate * 10000)):
                train_set.targets[i] += random.randint(1, args.num_classes - 1)
                train_set.targets[i] %= args.num_classes
    elif hasattr(train_set, "_labels"):
        #_labelsでラベル情報を持っている場合
        for i in range(len(train_set._labels)):
            if(random.randint(0, 9999) < int(label_noise_rate * 10000)):
                train_set._labels[i] += random.randint(1, args.num_classes - 1)
                train_set._labels[i] %= args.num_classes
    else:
        #labelsでラベル情報を持っている場合
        for i in range(len(train_set.labels)):
            if(random.randint(0, 9999) < int(label_noise_rate * 10000)):
                train_set.labels[i] += random.randint(1, args.num_classes - 1)
                train_set.labels[i] %= args.num_classes
    
    train_loader = torch.utils.data.DataLoader(train_set, batch_size=128, shuffle=True, num_workers=2)
    test_set = get_dataset_test(args, transform_test)
    test_loader = torch.utils.data.DataLoader(test_set, batch_size=128, shuffle=True, num_workers=2)
    # CIFAR10のクラス
    class_names = ('plane', 'car', 'bird', 'cat', 'deer', 'dog', 'frog', 'horse', 'ship', 'truck')
    
    model = get_model(args)
    model = model.to(device)
    print(args.model_fullname)

    if args.resize:
        args.dataset = "resize_" + args.dataset
        #resizeの有無をpath名で明示 

    # if device == 'cuda':
    #     model = torch.nn.DataParallel(model)

    criterion = nn.CrossEntropyLoss().to(device)
    optimizer = optim.Adam(model.parameters(), lr=0.0001)
    x1 = []
    x2 = range(epoch + 1)
    x1.append(0)

    #wandb

    wandb.init(
    # set the wandb project where this run will be logged
        project="DDvsShapeTexture",
        
        # track hyperparameters and run metadata
        config={
            "architecture": args.model_fullname,
            "epochs": args.epoch,
            "dataset": args.dataset,
            "label_noise_rate": args.label_noise_rate,
            "batch_size": args.batch_size,
            "learning_rate": args.learning_rate,
            "seed": args.fix_seed,
        }
    )


    #----------#
    ### csv 作成
    #----------#
    # 初期値取得
    
    train_acc, train_loss = test(model, device, test_loader, criterion) # 初期値取得
    test_acc, test_loss = test(model, device, test_loader, criterion)
    wandb.log({"Epoch": 0,"train acc": train_acc, "train loss": train_loss, "test acc": test_acc, "test loss": test_loss})

    # train
    with open('./csv/{}_{}_{}epochs_ln{}pc_seed{}_train.csv'.format(args.model_fullname, args.dataset, args.epoch, int(label_noise_rate * 100), args.fix_seed),'w') as file:
        file.write("epoch,error,loss\n")
        file.write(f"0,{1.0 - train_acc},{train_loss}" + "\n")
    # test
    with open('./csv/{}_{}_{}epochs_ln{}pc_seed{}_test.csv'.format(args.model_fullname, args.dataset, args.epoch, int(label_noise_rate * 100), args.fix_seed),'w') as file:
        file.write("epoch,error,loss\n")
        file.write(f"0,{1.0 - test_acc},{test_loss}" + "\n")

    if not os.path.isdir(f'./model_weight1/{args.model_fullname}_{args.dataset}_ln{int(label_noise_rate * 100)}pc_seed{args.fix_seed}'):
        os.makedirs(f'./model_weight1/{args.model_fullname}_{args.dataset}_ln{int(label_noise_rate * 100)}pc_seed{args.fix_seed}')
    ## ----------------#
    ## training #
    # ----------------#
    for epoch in range(epoch):
        # Train and test a model.
        model.train()
        #trainの各数値はbatchごとに出す
        #calc_scoreの返却値のlossはbatch数で割っているので無視,loss.item()を用いる
        for batich_idx, (inputs, targets) in enumerate(train_loader):
            output_list = []
            target_list = []
            running_loss = 0.0
            xpoint = 0.0 + epoch + (float(batich_idx + 1) / len(train_loader))
            inputs, targets = inputs.to(device), targets.to(device)
            outputs = model(inputs)
            loss = criterion(outputs, targets)

            optimizer.zero_grad()
            loss.backward()
            optimizer.step()

            output_list += [int(o.argmax()) for o in outputs]
            target_list += [int(t) for t in targets]
            running_loss += loss.item()

            train_acc, train_loss = calc_score(target_list, output_list, running_loss, train_loader)
            x1.append(xpoint)
            ## csv writer
            with open('./csv/{}_{}_{}epochs_ln{}pc_seed{}_train.csv'.format(args.model_fullname, args.dataset, args.epoch, int(label_noise_rate * 100), args.fix_seed),'a') as file:
                file.write(f"{xpoint},{1.0 - train_acc},{loss.item()}" + "\n")
            
            # if batich_idx % 100 == 0 and batich_idx != 0:
            #     stdout_temp = 'batch: {:>3}/{:<3}, train acc:{:<8}, train loss: {:<8}'
            #     print(stdout_temp.format(batich_idx, len(train_loader), train_acc, loss.item()))

        ## これないとダメmodel.eval
        model.eval()
        test_acc, test_loss = test(model, device, test_loader, criterion)
        # csv writer
        with open('./csv/{}_{}_{}epochs_ln{}pc_seed{}_test.csv'.format(args.model_fullname, args.dataset, args.epoch, int(label_noise_rate * 100), args.fix_seed),'a') as file:
                file.write(f"{epoch + 1},{1.0 - test_acc},{test_loss}" + "\n")

        # Output score.
        stdout_temp = 'epoch: {:>3}, train acc: {:<8}, train loss: {:<8}, test acc: {:<8}, test loss: {:<8}'
        print(stdout_temp.format(epoch+1, train_acc, loss.item(), test_acc, test_loss))
        wandb.log({"Epoch": epoch + 1,"train acc": train_acc, "train loss": loss.item(), "test acc": test_acc, "test loss": test_loss, "leraning rate": optimizer.param_groups[0]['lr']})

        ## model save###
        # torch.save(model.state_dict(), './model_weight/resnet18*' + str(args.model_width) + '-cifar10-train.csv')
        # pytorchの慣例でpthファイルで保存する
        # 実験のためにepochごとに保存
        model_path = f'./model_weight1/{args.model_fullname}_{args.dataset}_ln{int(label_noise_rate * 100)}pc_seed{args.fix_seed}/{args.model_fullname}_epoch{(epoch + 1):04}_{args.dataset}_ln{int(label_noise_rate * 100)}pc_seed{args.fix_seed}.pth'
        torch.save(model.state_dict(), model_path)
    
    # checkpoint
    torch.save({
    "epoch": epoch,
    "model_state_dict": model.state_dict(),
    "optimizer_state_dict": optimizer.state_dict()
    }, 
    f'./checkpoint/{args.model_fullname}_{(epoch + 1)}epochs_{args.dataset}_ln{int(label_noise_rate * 100)}pc_seed{args.fix_seed}.tar')
    
    

def train (model, device, train_loader, criterion, optimizer):
    model.train()
    output_list = []
    target_list = []
    running_loss = 0.0
    for batich_idx, (inputs, targets) in enumerate(train_loader):
        inputs, targets = inputs.to(device), targets.to(device)
        outputs = model(inputs)
        loss = criterion(outputs, targets)

        optimizer.zero_grad()
        loss.backward()
        optimizer.step()

        output_list += [int(o.argmax()) for o in outputs]
        target_list += [int(t) for t in targets]
        running_loss += loss.item()

        train_acc, train_loss = calc_score(target_list, output_list, running_loss, train_loader)
        if batich_idx % 100 == 0 and batich_idx != 0:
            stdout_temp = 'batch: {:>3}/{:<3}, train acc:{:<8}, train loss: {:<8}'
            print(stdout_temp.format(batich_idx, len(train_loader), train_acc, train_loss))
    train_acc, train_loss = calc_score(target_list, output_list, running_loss, train_loader)


    return train_acc, train_loss

def test(model, device, test_loader, criterion):
    model.eval()

    output_list = []
    target_list = []
    running_loss = 0.0
    with torch.no_grad():
        for batch_idx, (inputs, targets) in enumerate(test_loader):
            # Forward processing.
            inputs, targets = inputs.to(device), targets.to(device)
            outputs = model(inputs)
            loss = criterion(outputs, targets)

            # Set data to calculate score.
            output_list += [int(o.argmax()) for o in outputs]
            target_list += [int(t) for t in targets]
            running_loss += loss.item()

    test_acc, test_loss = calc_score(target_list, output_list, running_loss, test_loader)

    return test_acc, test_loss
def calc_score(true_list, predict_list, running_loss, data_loader):
    # import pdb;pdb.set_trace()
    # result = classification_report(true_list, predict_list, output_dict=True)
    # acc = round(result['accuracy'], 6)
    acc = accuracy_score(true_list, predict_list)
    loss = round(running_loss / len(data_loader), 6)

    return acc, loss

def parse_args():
    arg_parser = argparse.ArgumentParser(description="ResNet trained by CIFAR-10")
    
    arg_parser.add_argument("-k", "--model_width", type=int, default=64)
    arg_parser.add_argument("-e", "--epoch", type=int, default=4000)
    arg_parser.add_argument("-l", "--label_noise_rate", type=float, default=0.0)

    # 追加
    arg_parser.add_argument("--model", type=str, choices=["resnet18k", "resnet18", "resnet34", "resnet50", "densenet121", "mobilenetV2", "vitB16"], help="モデルアーキテクチャの選択")
    arg_parser.add_argument("-pt", "--pretrained", type=str, choices=["SR", "IN", "INv1", "INv2", "INv3", "INv4", "FR", "FDB1kv1", "RCDB1kv1"], default="IN", help="事前学習モデルの利用.指定しないと事前学習なし")
    arg_parser.add_argument("-ds", "--dataset", type=str, choices=["cifar10", "cifar100", "tinyImageNet", "DTD", "STL10", "cannyShapeCifar10ColorV1", "cifar100_to_10class"], default="cifar10")
    arg_parser.add_argument("--num_classes", type=int, default=10, help="分類クラス数")
    arg_parser.add_argument("--resize", action='store_true', help="224*224にresize,ViTはデフォルトでtrue")
    arg_parser.add_argument("--imagesize", type=int, help="resize=trueの場合指定サイズに変更")
    arg_parser.add_argument("-bs", "--batch_size", type=int, default=128)
    arg_parser.add_argument("-lr", "--learning_rate", type=float, default=0.0001)
    arg_parser.add_argument("-seed", "--fix_seed", type=int, default=42)

    return arg_parser.parse_args()

if __name__ =='__main__':
    warnings.filterwarnings('ignore')
    import time
    start = time.perf_counter()
    main()
    
    print(time.perf_counter() - start)

\end{minted}
}