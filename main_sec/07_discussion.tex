\chapter{考察}
本研究では,自然画像に存在する形状・テクスチャの特徴に着目し,二重下降現象との関係を分析した.その結果,ImageNetで事前学習を行った場合に,いくつかの条件において学習過程における二重降下とモデルが示す形状・テクスチャ偏重度の推移に相関がみられることがわかった.このような条件においては,二重降下における二度目の下降が始まるまでのテスト誤差と,形状・テクスチャ偏重度の推移との間に相関がみられる傾向にある.しかし,この相関は二度目の下降が始まると弱くなった.このような傾向は,二重降下を三つの段階に分割し,定性的,定量的に相関関係を評価することで判明した.

その後の実験では,CNNの最終畳み込み層で強く偏重度の独特なシフトが観察された.しかし,それ以前の中間層では,少なくとも検証を行った層においては,最終畳み込み層と同様の変化は見られなかった.この観察から,CNNの深い層は中間層とは異なる学習傾向を示す可能性が示唆される.

% 二重降下に関する先行研究では,データに存在する複数の特徴に影響されるのではないかという仮説が提唱されている.では,二重降下現象は形状やテクスチャーなどの特徴によって起こるのだろうか?我々はそうは考えていない.もし本当にそのような特徴によって現象が引き起こされるのであれば,二重降下の挙動と形状・テクスチャ偏重度は整理された傾向を示すはずである.例えば,まず形状偏重度がピークに達し,その後減少し,テクスチャ偏重度がピークに達する.しかし,実際には形状偏重度とテクスチャ偏重度それら自体が逆相関を見せている.また,事前学習を行わなかった場合には形状テクスチャ偏重度の特異な推移は見られなかった.したがって,double descentを引き起こす何らかの学習傾向によって,CNNの特徴抽出傾向が影響され,ImageNetで事前学習した場合において,double descentと偏重度の推移に相関を見せたと考える.(川勝先生の指摘あり,書き換え必須)
二重降下に関する先行研究では,データに存在する複数の特徴に影響されるのではないかという仮説が提唱されている.では,二重降下現象は形状やテクスチャーなどの特徴によって起こるのだろうか?我々は,その片鱗を観察したと考えている.今回の実験では,double descentを引き起こす何らかの学習傾向によって,CNNの特徴抽出傾向が影響され,ImageNetで事前学習した場合において,double descentと偏重度の推移に相関を見せたと考える.そのため,ImageNetで事前学習の有無で,パラメータの学習のされ方がどのように変わっているかを検証することは,二重降下を理解することにつながる可能性が考えられる.

実用的な観点からは,ImageNetを事前学習した条件下では,偏重度が最大,または最小となるepoch付近でテスト誤差が最大になる可能性が示唆され,この偏重度を観察することで,早期に停止できる最適,または準最適な学習エポック数を決定できる可能性が示唆される.さらに,二重降下を引き起こす要因が,特に深い層における形状やテクスチャの特徴に対するCNNの偏重度合いにも影響を与える可能性があることを示した.

本研究では,CNNが形状やテクスチャといった画像特徴をどのように学習していくかに着目し,複数の条件において二重降下との関係を検証した.深層学習において,未解明なことは多数存在し,double descentもその一つである.そのような中で,特に深層に目を向けるべきであるとした本研究は,今後の研究の一つの方向性を指示したと考える.しかし,今回見られた現象の具体的なメカニズムを示せていないことは一つの限界点である.
本研究では,CNNによる形状やテクスチャといった画像特徴の学習過程に注目し,さまざまな条件下における二重降下現象との関連性を検証した.深層学習における未解明の領域は多数存在し,二重降下現象はその一つである.本研究は,特に深層学習の深い層への理解を深めることの重要性を指摘し,将来の研究に対する一つの有望な方向性を提供するものと考えられる.しかしながら,観測された現象の具体的な機構に関しては明確な説明を提供できていない.この点は,今後の研究における主要な課題である.
\newpage