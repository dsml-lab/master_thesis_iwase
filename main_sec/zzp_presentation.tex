\chapter{対外成果発表リスト}
\begin{enumerate}
    \item[2022-03-15] \underline{\UTF{9AD9}橋 秀弥}, 鏡川 悠介,前田 英作, ``深層ニューラルネットワークにおける二重降下現象
,''  電子情報通信学会2022年総合大会 情報・システムソサイエティ特別企画 ジュニア&学生ポスターセッション予稿集, ISS-SP-028, ポスター発表, 開催地 zoom.
    \item[2022-04-16] 杉田 拓磨, 岡澤 律来, 金田 龍平, \underline{\UTF{9AD9}橋 秀弥}, 鏡川 悠介, 出場チーム名 AID, AIイノベーションアワード2022, 開催地 立教大学, \\最優秀賞(https://www.nttpc.co.jp/press/2022/04/202204211500.html).
    \item[2022-07-26] \underline{\UTF{9AD9}橋 秀弥}, 鏡川 悠介,前田 英作, ``深層学習における二重降下現象と画像のテクスチャ・形状性について
,''  第25回 画像の認識・理解シンポジウム(MIRU2022), OS1A-5, 口頭発表(査読あり),ポスター発表, 開催地 姫路市文化コンベンションセンター  アクリエひめじ(兵庫県).
    \item[2022-07-28] 杉田 拓磨, 岡澤 律来, 金田 龍平, \underline{\UTF{9AD9}橋 秀弥}, 鏡川 悠介, 前田 英作, ``物語文を入力とする自動挿絵生成システム
,''  第25回 画像の認識・理解シンポジウム(MIRU2022), IS3-75, ポスター発表, 開催地 姫路市文化コンベンションセンター  アクリエひめじ(兵庫県).
    \item[2023-03-02] \underline{\UTF{9AD9}橋 秀弥},井上中順,横田理央,片岡裕雄,前田英作, ``画像識別における形状・テクスチャ偏重度と二重降下現象の関係について
,''  パターン認識・メディア理解(PRMU)2023年3月研究会, PRMU-3, 口頭発表(査読なし), 公立はこだて未来大(北海道).
    \item[2023-03-02] 遠藤隆斗,\underline{\UTF{9AD9}橋 秀弥},前田英作, ``医療画像タスクにおける数式駆動型教師あり学習の有効性について
,''  パターン認識・メディア理解(PRMU)2023年3月研究会, PRMU-22, 口頭発表(査読なし), 公立はこだて未来大(北海道).
    \item[2023-07-28] \underline{\UTF{9AD9}橋 秀弥},井上中順,横田理央,片岡裕雄,前田英作, ``学習過程における形状・テクスチャ偏重度の推移と事前学習データセットとの関係について
,''  第26回 画像の認識・理解シンポジウム(MIRU2023), IS3-26,ポスター発表, 開催地 アクトシティ浜松(静岡県).




\end{enumerate}
